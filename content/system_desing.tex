%----------------------------------------------------------------------------
\chapter{\SystemDesign}
%----------------------------------------------------------------------------
\section{Követelmények}
%----------------------------------------------------------------------------

 

%----------------------------------------------------------------------------
\section{Rendszerterv}
%----------------------------------------------------------------------------


Az új rendszer tervezésekor a következő szempontokat tartottam szem előtt:








Hangrendszer szempontjából a Martin Audio Wavefront Precision szériás termékeire esett a választás.
Mivel teljes mértékben digitális rendszert elérése volt a cél, ezért a Dante modullal
rendelkező Martin Audio iK42 és iK81 végfokok tökéletesen illeszkedtek a rendszerbe.
Mindkét végfok csúcskategóriás teljesítményt és hangminőséget nyújt, és a D kategóriás
erősítőknek köszönhetően rendkívül kis helyet foglalnak el a rackben miközben kiváló hatásfok
mellett képesek nagy teljesítményt leadni. A végfokokat a Dante hálózaton keresztül
fogjuk jellel ellátni, így a hagyományos analóg XLR, vagy AES kábelezés helyett két CAT5E kábellel (a rendundancia miatt)
tudjuk a végfokokat a hálózatra kötni. A végfokrendszer fő vezérlő protokollja miatt szükség lesz még egy CAT5E alapú
összeköttetésre, ami az egyes végfokokat köti össze egy hálózatba a switcheken keresztül. (VU-NET protokoll)
Minden egyes végfokrackben két MikroTik 24 portos switch lesz. Az egyik switch a Dante elsődleges hálózatát fogja kizárólag kezelni. 
A másik switch a Dante másodlagos hálózatát, és a VU-NET hálózatot fogja kezelni. 
Ez a két alhálózat VLAN szegmensekkel lesz elkülönítve, hogy a hálózat biztonságos és stabil legyen.



!!!ÁBRA HELYE!!!

Egy jel útja a régi hibrid rendszerben


!!!ÁBRA HELYE!!!

Egy jel útja az új teljesen digitális rendszerben

%----------------------------------------------------------------------------
\subsection{Martin Audio Wavefront Precision hangrendszer}
%----------------------------------------------------------------------------
\subsubsection{Martin Audio Display 2.3.4 b1 tervező szoftver}
%----------------------------------------------------------------------------
Mielőtt bele kezdenénk a tervezési folyatba, fontos megemlíteni, hogy a szoftver 
eredetileg Intel alapú processzorokra lett tervezve és MatLab alapú. Ebből fakadóan
AMD Ryzen processzorokon habár elidult a szoftver, de nem volt stabil és a számítások során
minden esetben összeomlott, és használtatalanul lassú volt. Személy szerint a saját gépem amivel dolgoztam
sajnos ilyen processzorral van szerelve ezért muszáj volt megoldást találni a problémára.
A Martin Audio hivatalos szoftveres támogatásához fordultam először, de sajnos nem tudtak segíteni.
Ezért a szoftver használatához
sok belefektetett óra olvasás után sikerült egy olyan MatLab CMD parancsot találnom, amivel
a szoftver elindul és használható. 
Miután rájöttem a probléma gyökerére, ezt megosztottam velük, hogy a jövőben másoknak ne kelljen
ezzel a problémával szembesülniük. 
A hiba az alábbi volt. Az új AMD Ryzen processzorok másfajta utasításkészletet használnak. 
Ebből kifolyólag a MatLab 2015-s runtime alapú szoftver adta alaputasításokat nem tudta értelmezni a CPU.
A vezető szoftvermérnökkel való e-mail-es beszélgetésünk során megköszönte a probléma
megoldását, és nemsokkal a megoldásom megosztása után a hivatalos oldalra feltölttötték
az indító parancsfáljt. Az e-mailben további kollaborációra is adott lehetőséget.
A kompatibilási problémát rögtön a script elején megoldottam,
mivel a következő parancs megadásával már használhatóvá válik a program: \texttt{set MKL\_DEBUG\_CPU\_TYPE=5} \newline
Ez a sor a program vezérlését AVX2-re állítja át, és mivel ezt az utasításkészletet már ismeri az AMD Ryzen processzor
is ezért a probléma már a múlté.
Az indító fájl további sorai optimalizálások a számítások gyorsítására, és a párhuzamosítására, ezzel jobban kihasználva
a rendelkezésre álló hardver erőforrásokat.
%----------------------------------------------------------------------------
\begin{lstlisting}[caption={A Display 2.3.4 b1 indító ".bat" scriptje AMD Ryzen processzorokhoz}, label=batcode, xleftmargin=\parindent]
    @echo off
    set PATH=%PATH%;C:\Program Files\Martin Audio\Display2_3_4_b1\application
    set MKL_DEBUG_CPU_TYPE=5
    set options=optimoptions('ga','UseParallel',true,'UseVectorized',false)
    set options=optimoptions('gamultiobj','UseParallel',true,'UseVectorized',false)
    set options=optimoptions('paretosearch','UseParallel',true)
    set options=optimoptions('particleswarm','UseParallel',true,'UseVectorized',false)
    set options=optimoptions('patternsearch','UseParallel',true,'UseCompletePoll',true,'UseVectorized',false)
    set options=optimoptions('surrogateopt','UseParallel',true)
    set GPUAcceleration=on
    start "Martin Audio" Display2_3_4_b1.exe
    pause
\end{lstlisting}
%----------------------------------------------------------------------------
\begin{figure}[H]
    \centering
    \includegraphics[width=\textwidth, keepaspectratio]{figures/ambrose_email.png}
    \caption{E-mail a Martin Audio vezető szoftvermérnökétől}
    \label{fig:ambrose_email}
\end{figure}
%----------------------------------------------------------------------------
Most, hogy már a szoftver használható és teljes mértékben működőképes, kezdjük el a tervezést.
A modellezés során a budapesti Millenáris B csarnoka lesz a referencia helyszín. Két LineArray rendszert fogunk
tervezni, mivel a terem hosszúsága és a lefedettség növelése miatt szükségünk lesz Delay kiegészítésre a fő hangrendszerhez.
Első lépésben a fő hangrendszert tervezem meg, ami oldalanként (bal és jobb) 8 darab WPC LineArray modulból fog állni.
Ez a láda 2 darab 10"-os mélysugárzót, 2 darab 5"-os közép sugárzót és 4 darab 0.7"-os magassugárzót tartalmaz.
Három utas Bi-amp meghajtású külső végfokot igénylő rendszer, ahol a mély tartományt (+1,-1) és a középmagas tartományt (+2,-2) külön kezeljük,
a négypólusú Neutrik Speakon csatlakozókon keresztül. 
A láda maximális hangnyomás szintje 135 dB, és 65 Hz-től 18 kHz-ig terjed a frekvencia átvitele +- 3 dB pontossággal. \cite{MARTINAUDIOWPC}
%----------------------------------------------------------------------------
\begin{figure}[H]
    \centering
    \includegraphics[width=80mm, keepaspectratio]{figures/wpc_front_view.jpg}
    \caption{Martin Audio WPC LineArray modul}\label{fig:wpc}
\end{figure}
%----------------------------------------------------------------------------
A program megnyitásakor a legelső lépés, hogy kiválasztjuk a termékpalettából a megfelelő hangrendszert. 
Jelen esetben az előbbiekben említett WPC-t. A produkció igényei, a nagylétszámú közönség és a ládamennyiség miatt a rendszert
\textit{``riggelni''} fogjuk. (maximálisan 6 darab WPC-t lehet \textit{``stackelni''}, azaz a földre vagy mélyládákra helyezni)
A helyszín felmérése után a hangrendszer \textit{``riggelése''} lehetséges, mivel a csarnokban található tartószerkezet biztonságosan
és tartósan képes elviselni a rendszer súlyát.
A telepítés módja kiválasztása után megadjuk a szoftvernek a tervezni való hangláda mennyiséget, ez az esetünkben már említett 8 darab.
A hozzáadás gombra kattintva a elénk kerül a fő kezelőfelület, ahol a hangrendszert tudjuk lépésről lépésre tervezni.
%----------------------------------------------------------------------------
\begin{figure}[H]
    \centering
    \includegraphics[width=50mm, keepaspectratio]{figures/display_wpc_0.png}
    \caption{Display 2.3.4 b1 szoftver kezdőképernyője}\label{fig:display_wpc_0}
\end{figure}
%----------------------------------------------------------------------------
A tervezési folyamat öt alrészre osztható, amiket a szoftverben külön kezelünk.
Ezeket a \textit{``Slice''}, \textit{``Cover''}, \textit{``Splay''}, \textit{``Rig''} és \textit{``EQ''} kezelőfelületeken tudjuk elvégezni,
balról jobbra haladva. Mivel a különböző részegységek egymásra épülnek, ezért fontos a sorrend betartása.
(tervezés utáni módosításokra természetesen van lehetőség, de az adott projekt első tervezési folyamata során ezeket a lépéseket kell követni)
%----------------------------------------------------------------------------
\begin{figure}[H]
    \centering
    \includegraphics[width=\textwidth, keepaspectratio]{figures/display_wpc_0_1.png}
    \caption{Display 2.3.4 b1 szoftver fő kezelőfelülete}\label{fig:display_wpc_0_1}
\end{figure}
%----------------------------------------------------------------------------
A \textit{``Slice''} panelen meghatározzuk a rendszer fizikai pozícióját térben. A csarnok
pontos lemodellezése érdekében a mérésekhez lézeres távolságmérőt használtam.
Mivel minden egyes rendezvényen más és más a különböző elemek elhelyezkedése, ezért a
rendszert minden alkalommal újra kell tervezni, még akkor is ha maga a helyszín nem változik.
\textit{``Vertex''} pontok segítségével tudjuk a méreteket és a pozíciókat meghatározni.
A 2D-s modellen figyelembe kell venni a terem önálló méretén kívül a színpadod és a színpad mögötti területet is.
A rajznak tartalmazni kell azokat a falfelületeket is amelyeknél a hangvisszaverődést minimalizálni szeretnénk,
ennek az optimalizáció későbbi fázisában lesz jelentősége.
A terem pontos rajza után még két fontos paramétert kell megadni ezen a felületen.
El kell helyeznünk magát a hangrendszert a teremben, és meg kell határoznunk milyen magasra szeretnénk a rendszert emelni.
Mivel a csarnok rendkívül hosszú, és a adottságai megengedik, ezért a rendszert minél magasabbra szeretnénk emelni, 
a jobb lefedettség érdekében.


%----------------------------------------------------------------------------
\begin{figure}[H]
    \centering
    \includegraphics[width=\textwidth, keepaspectratio]{figures/display_wpc_1.png}
    \caption{Display 2.3.4 b1 szoftver \textit{``Slice''} kezelőfelülete}\label{fig:display_wpc_1}
\end{figure}
%----------------------------------------------------------------------------

%----------------------------------------------------------------------------
\begin{figure}[H]
    \centering
    \includegraphics[width=\textwidth, keepaspectratio]{figures/display_wpc_2.png}
    \caption{Display 2.3.4 b1 szoftver \textit{``Cover''}kezelőfelülete}\label{fig:display_wpc_2}
\end{figure}
%----------------------------------------------------------------------------

%----------------------------------------------------------------------------
\begin{figure}[H]
    \centering
    \includegraphics[width=\textwidth, keepaspectratio]{figures/display_wpc_3.png}
    \caption{Display 2.3.4 b1 szoftver \textit{``Splay''}kezelőfelülete}\label{fig:display_wpc_3}
\end{figure}
%----------------------------------------------------------------------------

%----------------------------------------------------------------------------
\begin{figure}[H]
    \centering
    \includegraphics[width=\textwidth, keepaspectratio]{figures/display_wpc_4.png}
    \caption{Display 2.3.4 b1 szoftver \textit{``Rig''}kezelőfelülete}\label{fig:display_wpc_4}
\end{figure}
%----------------------------------------------------------------------------

%----------------------------------------------------------------------------
\begin{figure}[H]
    \centering
    \includegraphics[width=\textwidth, keepaspectratio]{figures/display_wpc_5.png}
    \caption{Display 2.3.4 b1 szoftver \textit{``EQ''}kezelőfelülete}\label{fig:display_wpc_5}
\end{figure}
%----------------------------------------------------------------------------

%----------------------------------------------------------------------------
\begin{figure}[H]
    \centering
    \includegraphics[width=\textwidth, keepaspectratio]{figures/display_wpc_6.png}
    \caption{Display 2.3.4 b1 szoftver \textit{``SPL''}kezelőfelülete}\label{fig:display_wpc_6}
\end{figure}
%----------------------------------------------------------------------------

%----------------------------------------------------------------------------
\begin{figure}[H]
    \centering
    \includegraphics[width=\textwidth, keepaspectratio]{figures/display_wpc_7.png}
    \caption{Display 2.3.4 b1 szoftver exportáló kezelőfelülete}\label{fig:display_wpc_7}
\end{figure}
%----------------------------------------------------------------------------

%----------------------------------------------------------------------------
\subsubsection{Martin Audio VU-NET rendszer szoftver}
%----------------------------------------------------------------------------


%----------------------------------------------------------------------------
\subsection{Allen \& Heath digitális keverőrendszer}
%----------------------------------------------------------------------------



%----------------------------------------------------------------------------
\subsection{Shure ULXD digitális vezeték nélküli mikrofonrendszer}
%----------------------------------------------------------------------------



%----------------------------------------------------------------------------
\subsection{Dante audio szerver}
%----------------------------------------------------------------------------



%----------------------------------------------------------------------------
\subsection{Dante hálózat kialakítása és optimalizálása}
%----------------------------------------------------------------------------
\subsubsection{Dante Controller: Hálózati mátrix}
%----------------------------------------------------------------------------
Ezen a felületen tudjuk a hálózaton összekapcsolni a különböző hang vevőket és
adókat. Egy nagyobb rendszerben a konfigurálása rendkívül nagy odafigyelést és
precíziót igényel, pontosan tudnunk kell mit, hogyan és miért kötünk össze.
%----------------------------------------------------------------------------
\subsubsection{Dante Controller: Eszköz nézet}
%----------------------------------------------------------------------------
Mielőtt elkezdenénk konfigurálni az adott eszközt, fontos eldöntenünk, hogy
milyen módban szeretnénk használni.
Lehetőségünk van két fő mód közül választani, a redundáns és a
váltott mód közül. A redundant mód mint ahogy azt a neve is sugallja
redundáns kommunikációt valósít meg az eszközök között szoftveresen és
hardveresen egyaránt. Az összes Dante kártya a jelenlegi rendszerben gyárilag két RJ45-s
csatlakozóval rendelkezik. Jelen esetben ezt a módot választjuk az
üzembiztosság és a kritikus hibák minimalizálása miatt.
A másik lehetőség a váltott pedig eszközök láncolását
teszi egyszerűbbé. Amennyiben a rendundancia nem elsődleges szempont számunkra, nem kell
minden egyes eszköz mögé switch, hanem a másodlagos RJ45 port direktbe köti
az arra csatlakoztatott eszközt az elsődleges hálózatra. Így gyorsabban és
költséghatékonyabban tudjuk kiépíteni a hálózatot, azonban a redundancia lehetősége megszűnik. 
%----------------------------------------------------------------------------
\subsubsection{IP kiosztás}
%----------------------------------------------------------------------------
A rendszer képes automatikusan IP címeket osztani az egyes eszközöknek, 
így meggyorsítva a munkafolyamatot.
Azonban egy fixen előre megtervezett rendszernél praktikusabb és
üzembiztosabb megoldás, ha minden eszköznek manuálisan megadjuk a címét a
hálózaton. A tervezett rendszerben minden egyes eszköznek fix IP címet adtam,
hogy könnyen és logikusan átlátható legyen az előbb említett előnyökön kívül.
A címeket egy online is elérhető Excel táblázatban tároltam, hogy amennyiben szükség van rá
bármikor könnyen elérhető legyen. Ez a táblázat a cégnél dolgozó összes munkatárs számára látható,
aki a rendszerrel foglalkozik. Így amennyiben új eszköz kerül a hálózatra, vagy egy eszköz IP címét
valamilyen okból meg kell változtatni, egyszerűen elérhető a szükséges információ.

%----------------------------------------------------------------------------
\begin{figure}[H]
    \centering
    \includegraphics[width=\textwidth, keepaspectratio]{figures/dante_ips.png}
    \caption{Dante eszközök IP címei a hálózaton}\label{fig:dante_ips}
\end{figure}
%----------------------------------------------------------------------------





%----------------------------------------------------------------------------
\subsubsection{Dante Controller: Órajel nézet}
%----------------------------------------------------------------------------
Meg kell adnunk az audio hálózatunk master órajelét, ehhez az órajelhez
szinkronizál a többi eszköz, az időszinkronizáció kulcsontosságú élőzenei
produkcióknál.



%----------------------------------------------------------------------------
\section{Rendszermérések és monitorozás}
%----------------------------------------------------------------------------



%----------------------------------------------------------------------------
\subsection{Dante rendszer monitorozása}
%----------------------------------------------------------------------------



%----------------------------------------------------------------------------
\subsubsection{Dante Controller: Hálózati állapot nézet}
%----------------------------------------------------------------------------



%----------------------------------------------------------------------------
\subsubsection{Dante Controller: Események nézet}
%----------------------------------------------------------------------------



%----------------------------------------------------------------------------
\subsection{Cardioid mélyláda rendszer mérése}
%----------------------------------------------------------------------------



%----------------------------------------------------------------------------
\subsection{Mélyláda és Line Array fázishelyesség}
%----------------------------------------------------------------------------



%----------------------------------------------------------------------------
\subsection{Rendszer hangnyomás szint és frekvencia átvitel mérése}
%----------------------------------------------------------------------------







