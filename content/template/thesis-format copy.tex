%----------------------------------------------------------------------------
\chapter{A dolgozatról}
%----------------------------------------------------------------------------

%----------------------------------------------------------------------------
\section{A dolgozat célja}
%----------------------------------------------------------------------------
A szakdolgozat és a diplomamunka célja annak bizonyítása, hogy a jelölt önálló mérnöki munkára képes. Az elkészített mű tehát saját alkotómunkát kell, hogy bizonyítson!

%----------------------------------------------------------------------------
\section{A dolgozat felépítése}
%----------------------------------------------------------------------------
A dolgozat sorszámozott fejezetekből, illetve alfejezetekből áll. A tartalomjegyzék a sablon szerint a dolgozat elején legyen. A dolgozat néhány oldalas bevezetővel kezdődjék, amely bemutatja a feldolgozott szakterületet, ha szükséges történelmi utalásokat is tehet, s a jelölt itt jusson el a megoldandó probléma világos, tényszerű megfogalmazásához, s vázolja fel, hogy azt milyen módon oldotta meg. Ha szükséges, röviden bemutathatja a céget, ahol a munkát végezte, méltathatja a felvetett probléma időszerűségét, a megoldás korszerűségét. A bevezető célja, hogy a bíráló, vagy az olvasó el tudja helyezni az elkészített munkát a szakmán belül. A dolgozat fejezeteinek számozását a Bevezető fejezettel kezdje.

A felhasznált szakirodalomnak a jelölt által történő feldolgozása és bemutatása rendkívül fontos, hiszen a munkát arra alapozva készíti el. Szükséges tehát egy olyan fejezet megírása is (amely rendre a bevezetést követi), amelyben a jelölt a szakirodalomra (szakkönyvek, szakcikkek, tankönyvek) hivatkozva összegzi a már ismert tényeket, eredményeket és összefüggéseket. A felhasznált irodalom feldolgozásáról szóló fejezet ne a jól ismert tananyag ismétlése legyen! Törekedjen arra, hogy a fejezet áttekintése után az olvasó elegendő ismerettel rendelkezzen ahhoz, hogy a jelölt saját munkáját megértse. Használja a könyvtárat, s válogasson az interneten közzétett anyagok között, de kerülje a kétes eredetű forrásokat! Ez a fejezet kb. 10-15 oldal terjedelmű legyen.

A következő fejezet az elvégzett munkát hivatott bemutatni, s így a terjedelme is nagyobb kell, hogy legyen. A dolgozat írásakor ezen fejezetben nyugodtan használhat egyes szám első személyt (pl. megoldottam, megterveztem stb.), hiszen a munka a sajátja. A fejezetet célszerűen a feladat részletes leírásával kezdje, térjen ki minden lényeges momentumra. Gyakran előfordul, hogy a feladat egy meglevő rendszer átalakítása, bővítése. Ilyenkor a meglevő rendszer ismertetése a fejezet elején történjen meg, a változtatások bemutatása, a tervezés menete, az elvégzett lépések indoklása stb. pedig a fejezet fő súlypontját alkossák. Ebben a fejezetben fotókon, ábrákon, grafikonokon, képleteken keresztül érthetően, világosan mutassa be, hogy mi a saját, önálló tevékenysége, mit és hogyan oldott meg, azokból milyen eredmények születtek. A fejezet végén elemezze az elkészült munkát. Ez a fejezet legyen kb. 30-40 oldal, s ez legyen a dolgozat hangsúlyos része. Amennyiben munkája több, határozottan elkülönülő tevékenységből állt, ez a rész több fejezetre is tagolható.

A dolgozatot az összefoglalás zárja. Itt múlt időben a szerző röviden ismételje meg, hogy mit és hogy valósított meg. Ebben a rövid, egy-két oldalas fejezetben a jelölt rámutathat a még megoldásra váró kérdésekre, esetleges jövőbeni tervekre, feladatokra. Írja le tapasztalatait, következtetéseit.

A következő szakasz az irodalomjegyzék, amelynek formája kötött. A Tanszék megkötése, hogy az internetes források száma nem érheti el a teljes irodalmi hivatkozások számának 30\%-át, továbbá internetes forrás megjelölésekor kötelező a honlap utolsó látogatásának időpontját is megadni! Az irodalmi hivatkozásokat a szövegben a megfelelő helyen jelölni kell. A szerzők nevét mindenütt “Családnév, X.” formában kell megadni, ahol X. a szerző keresztnevének (keresztneveinek) kezdőbetűje. Magyar cikk esetén a vessző a családnév és a keresztnév kezdőbetűje közt elhagyható. Ha az egyértelműség megkívánja, a keresztnév kiírható teljesen is. Az irodalmi hivatkozások \ref{sec:HowtoReference} fejezetben bővebben kitérünk.

%----------------------------------------------------------------------------
\section{Formai követelmények}
%----------------------------------------------------------------------------
A \LaTeX{} sablon előnye, hogy ezzel nem kell foglalkoznod. Ha rendeltetésszerűen használod a sablont, akkor formai szempontból a dolgozat megfelelő lesz.

%TODO@FMA: SZE követelmenyek
%----------------------------------------------------------------------------
\section{A dolgozat nyelve}
%----------------------------------------------------------------------------
Mivel Magyarországon a hivatalos nyelv a magyar, ezért alapértelmezésben magyarul kell megírni a dolgozatot. Aki külföldi posztgraduális képzésben akar részt venni, nemzetközi szintű tudományos kutatást szeretne végezni, vagy multinacionális cégnél akar elhelyezkedni, annak célszerű angolul megírnia diplomadolgozatát. Mielőtt a hallgató az angol nyelvű verzió mellett dönt, erősen ajánlott mérlegelni, hogy ez mennyi többletmunkát fog a hallgatónak jelenteni fogalmazás és nyelvhelyesség terén, valamint -- nem utolsó sorban -- hogy ez mennyi többletmunkát fog jelenteni a konzulens illetve bíráló számára. Egy nehezen olvasható, netalán érthetetlen szöveg teher minden játékos számára.

%TODO@FMA: SZE követelmenyek
%----------------------------------------------------------------------------
\section{A dokumentum nyomdatechnikai kivitele}
%----------------------------------------------------------------------------
A dolgozatot A4-es fehér lapra nyomtatva, 2,5 centiméteres margóval (+1~cm kötésbeni), 11--12 pontos betűmérettel, talpas betűtípussal és másfeles sorközzel célszerű elkészíteni.

Annak érdekében, hogy a dolgozat külsőleg is igényes munka benyomását keltse, érdemes figyelni az alapvető tipográfiai szabályok betartására~\cite{Jeney}.
