\pagenumbering{roman}
\setcounter{page}{1}

\selecthungarian

%----------------------------------------------------------------------------
% Kivonat Magyarul 
%----------------------------------------------------------------------------
\chapter*{Kivonat}
% TODO: Távolítsd el a megjegyzést, ha mégis szeretnéd, hogy bekerüljön a tartalomjegyzékbe
%\addcontentsline{toc}{chapter}{Kivonat}

Szakdolgozatom célja egy élőzenei produkció hangrendszerének megtervezése, 
kiépítése, optimalizálása és beüzemelése, mely a Dante protokollra 
építkezik. Az Audio over IP rendszerek, különösen a Dante hálózatok, 
lehetőséget nyújtanak a hagyományos analóg hangrendszerekhez 
képest gyorsabb, megbízhatóbb és skálázhatóbb megoldások alkalmazására. 

Az 1. fejezet bevezeti a szakdolgozat témáját és eredetét,
miért is ezt a témát választottam.

A 2. fejezet az alapvető hangtechnikai fogalmakat és rendszereket 
mutatja be, beleértve az analóg és digitális hangrendszerek 
közötti eltéréseket. 

A 3. fejezet célja, hogy alaposan bemutassa a digitális audió 
rendszerek alapjait és a Dante protokoll technikai részleteit. 
Ebben a részben többek között bemutatásra kerülnek az IP-címek kiosztásának módszerei, 
a hálózati topológiák kialakítása, 
valamint a különböző hálózati eszközök, mint például a switchek szerepei.

A 4. fejezet a tervezési és telepítési folyamatot tárgyalja, 
különös figyelmet szentelve a felhasznált eszközöknek, 
mint például a hangprocesszorok, végerősítők és 
hangszórók. 
A telepítés során a Dante alapú hálózat  kiépítése mellett kiemelten
foglalkozom a hangminőség optimalizálásával. A valós környezetben 
végzett tesztelés során a rendszer megbízhatóságát és 
rugalmasságát is értékelem.

Az 5. fejezet a telepítés utáni üzemeltetési tapasztalatokkal 
foglalkozik. Részletesen bemutatom a rendszer integrálását a 
TéDé Rendezvények egy élőzenei produkcióján, ahol 
valós időben teszteltem a Dante hálózat teljesítményét. 
Itt tárgyalom a hangrendszer és a vezérlőrendszerek 
integrációjának fontosságát, valamint az előadás közben 
tapasztalt akusztikai és hálózati kihívások kezelését. 
Az eredmények alapján további fejlesztési lehetőségeket 
javaslok, melyek még tovább növelhetik a rendszer teljesítményét és rugalmasságát.

Az általam bemutatott rendszer lényeges előnyöket nyújt egy analóg 
hangrendszerrel szemben.
Az üzemeltetési tapasztalatok azt igazolják, hogy 
az Audio over IP technológiák, különösen a Dante, 
stabil, skálázható és megbízható megoldást 
kínál a modern élőzenei produkciók számára.


\vfill
\selectenglish


%----------------------------------------------------------------------------
% Abstract in English
%----------------------------------------------------------------------------
\chapter*{Abstract}
% TODO: Távolítsd el a megjegyzést, ha mégis szeretnéd, hogy bekerüljön a tartalomjegyzékbe
%\addcontentsline{toc}{chapter}{Abstract}

The goal of my thesis is to design, build, optimize, and commission a sound system for 
a live music production, based on the Dante protocol. Audio over IP systems, 
particularly Dante networks, offer faster, more reliable, and scalable 
solutions compared to traditional analog sound systems.

Chapter 1 introduces the topic and origin of the thesis, explaining why I chose this subject.

Chapter 2 presents the basic concepts and systems of sound engineering, including 
the differences between analog and digital sound systems.

Chapter 3 aims to thoroughly explain the fundamentals of digital audio systems and 
the technical details of the Dante protocol. This chapter covers methods for IP 
address allocation, network topology design, and the roles of various network devices, such as switches.

Chapter 4 discusses the design and installation process, with a particular focus on 
the equipment used, such as audio processors, power amplifiers, and loudspeakers. 
In addition to the installation of the Dante-based network, special attention is given to optimizing sound quality. The reliability and flexibility of the system are also evaluated during testing in a real-world environment.

Chapter 5 addresses the operational experiences post-installation. I provide a 
detailed account of integrating the system into a live music production 
by TéDé Rendezvények, where I tested the performance of the Dante network in real time. 
This chapter also discusses the importance of integrating sound systems and 
control systems, as well as addressing acoustic and network challenges 
encountered during the performance. Based on the results, I suggest further 
improvements that could enhance the system's performance and flexibility.

The system I present offers significant advantages over analog sound systems. 
Operational experiences confirm that Audio over IP technologies, particularly Dante, provide a stable, scalable, and reliable solution for modern live music productions.

\vfill
\selectthesislanguage

\newcounter{romanPage}
\setcounter{romanPage}{\value{page}}
\stepcounter{romanPage}