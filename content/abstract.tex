\pagenumbering{roman}
\setcounter{page}{1}

\selecthungarian

%----------------------------------------------------------------------------
% Kivonat Magyarul 
%----------------------------------------------------------------------------
\chapter*{Kivonat}
% TODO: Távolítsd el a megjegyzést, ha mégis szeretnéd, hogy bekerüljön a tartalomjegyzékbe
%\addcontentsline{toc}{chapter}{Kivonat}

Szakdolgozatomban egy olyan digitális hangtechnikai rendszer tervezését és megvalósítását mutatom be, amely teljes mértékben digitális alapokra helyezi a hangsúlyt.
Be fogom mutatni a rendszer tervezésének lépéseit, a különböző protokollok közötti választást, a rendszer felépítését, és a rendszer működését.


\vfill
\selectenglish


%----------------------------------------------------------------------------
% Abstract in English
%----------------------------------------------------------------------------
\chapter*{Abstract}
% TODO: Távolítsd el a megjegyzést, ha mégis szeretnéd, hogy bekerüljön a tartalomjegyzékbe
%\addcontentsline{toc}{chapter}{Abstract}

This document is a \LaTeX-based skeleton for BSc/MSc~theses based on the official template developed and maintained at the Electrical Engineering and Informatics Faculty, Budapest University of Technology and Economics. The goal of this skeleton is to guide and help students that wish to use \LaTeX{} for their work at \sze{} \givk. It has been tested with the \emph{TeXLive} \TeX~implementation, and it requires the PDF-\LaTeX~compiler.

Many thanks to the Fault Tolerant Systems Research Group who maintain the repository this template is based on: \url{https://github.com/FTSRG/thesis-template-latex}


\vfill
\selectthesislanguage

\newcounter{romanPage}
\setcounter{romanPage}{\value{page}}
\stepcounter{romanPage}