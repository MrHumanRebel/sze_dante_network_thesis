\pagenumbering{roman}
\setcounter{page}{1}

\selecthungarian

%----------------------------------------------------------------------------
% Kivonat Magyarul 
%----------------------------------------------------------------------------
\chapter*{Kivonat}
% TODO: Távolítsd el a megjegyzést, ha mégis szeretnéd, hogy bekerüljön a tartalomjegyzékbe
%\addcontentsline{toc}{chapter}{Kivonat}

Szakdolgozatom célja egy élőzenei produkció hangrendszerének megtervezése, 
kiépítése, optimalizálása és beüzemelése, mely a Dante protokollra 
építkezik. Az Audio over IP rendszerek, különösen a Dante hálózatok, 
lehetőséget nyújtanak a hagyományos analóg hangrendszerekhez 
képes gyorsabb, megbízhatóbb és skálázhatóbb megoldások alkalmazására. 
A Dante protokoll technológiai előnyei közé tartozik az alacsony 
késleltetés, a magas szintű jelminőség és a hálózati redundancia 
biztosítása, ami különösen fontos élő produkciók során.

Az 1. fejezet bevezeti a szakdolgozat témáját és eredetét,
miért is ezt a témát választottam.

A 2. fejezet az alapvető hangtechnikai fogalmakat és rendszereket 
mutatja be, beleértve az analóg és digitális hangrendszerek 
közötti eltéréseket. Ismerteti a digitális audiojelek 
feldolgozásának alapjait, kitér a hangerősségre, a hangnyomásszintre, valamint a 
különböző hangforrások közötti eltérésekre. 

A 3. fejezet célja, hogy alaposan bemutassa a digitális audió 
rendszerek alapjait és a Dante protokoll technikai részleteit. 
Ebben a részben többek között bemutatásra kerülnek az IP-címek kiosztásának módszerei, 
a hálózati topológiák kialakítása, 
valamint a különböző hálózati eszközök, mint például a switchek szerepei.

A 4. fejezet a tervezési és telepítési folyamatot tárgyalja, 
különös figyelmet szentelve a felhasznált eszközöknek, 
mint például a hangprocesszorok, végerősítők és 
hangszórók. A tervezési szakasz kulcsfontosságú eleme a
rendszer méretezése és konfigurálása, amely figyelembe 
veszi a produkció követelményeit, a helyszín 
méretét, a közönség elrendezését és az előadás 
specifikus igényeit. Ezen felül bemutatom a Dante 
Controller szoftver használatát, amely lehetővé teszi a 
hálózati eszközök konfigurálását és a forgalom 
monitorozását. 
A telepítés során a Dante alapú hálózat  kiépítése mellett kiemelten
foglalkozom a hangminőség optimalizálásával. A valós környezetben 
végzett tesztelés során a rendszer megbízhatóságát és 
rugalmasságát is értékelem.

Az 5. fejezet a telepítés utáni üzemeltetési tapasztalatokkal 
foglalkozik. Részletesen bemutatom a rendszer integrálását a 
TéDé Rendezvények egy élőzenei produkcióján, ahol 
valós időben teszteltem a Dante hálózat teljesítményét. 
Itt tárgyalom a hangrendszer és a vezérlőrendszerek 
integrációjának fontosságát, valamint az előadás közben 
tapasztalt akusztikai és hálózati kihívások kezelését. 
Az eredmények alapján további fejlesztési lehetőségeket 
javaslok, melyek még tovább növelhetik a rendszer teljesítményét és rugalmasságát.

Az általam bemutatott rendszer lényeges előnyöket nyújt egy analóg 
hangrendszerekkel szemben.
Az üzemeltetési tapasztalatok azt igazolják, hogy 
az Audio over IP technológiák, különösen a Dante, 
stabil, skálázható és megbízható megoldást 
kínál a modern élőzenei produkciók számára.


\vfill
\selectenglish


%----------------------------------------------------------------------------
% Abstract in English
%----------------------------------------------------------------------------
\chapter*{Abstract}
% TODO: Távolítsd el a megjegyzést, ha mégis szeretnéd, hogy bekerüljön a tartalomjegyzékbe
%\addcontentsline{toc}{chapter}{Abstract}

The aim of my thesis is to design, build, optimize, and deploy a sound system 
for a live music production, based on the Dante protocol. 
Audio over IP systems, particularly Dante networks, enable the use 
of faster, more reliable, and scalable solutions compared to 
traditional analog sound systems. Among the technological 
advantages of the Dante protocol are low latency, high signal quality, 
and network redundancy, which are crucial for live productions.

Chapter 1 introduces the topic and origin of the thesis, 
explaining why I chose this subject.

Chapter 2 presents the basic concepts and systems of audio technology, 
including the differences between analog and digital sound systems. 
It covers the fundamentals of digital audio signal processing, 
addressing aspects like volume, sound pressure levels, and differences 
among various sound sources.

Chapter 3 thoroughly explores the foundations of digital audio systems 
and the technical details of the Dante protocol. This section includes 
methods for IP address allocation, network topology design, and the 
roles of various network devices such as switches.

Chapter 4 discusses the design and installation process, with special 
attention to the equipment used, such as audio processors, amplifiers, and speakers. 
A critical aspect of the design phase is the sizing and configuration 
of the system, considering the production requirements, venue size, 
audience layout, and specific needs of the performance. 
Additionally, I demonstrate the use of Dante Controller software for 
configuring network devices and monitoring traffic. During installation, 
I focus on building a Dante-based network while optimizing sound quality. 
The testing phase evaluates the reliability and flexibility of the system in a real-world environment.

Chapter 5 addresses the operational experiences after installation. 
It provides a detailed account of integrating the system into a 
live music production by TéDé Rendezvények, where I tested the performance 
of the Dante network in real-time. This chapter discusses the importance of 
integrating the sound and control systems and how I handled acoustic and 
network challenges during the performance. Based on the results, 
I propose further development opportunities to enhance the system’s performance and flexibility.

The system presented in my thesis offers significant advantages over analog sound systems. 
The operational experiences confirm that Audio over IP technologies, especially Dante, 
provide a stable, scalable, and reliable solution for modern live music productions.

\vfill
\selectthesislanguage

\newcounter{romanPage}
\setcounter{romanPage}{\value{page}}
\stepcounter{romanPage}