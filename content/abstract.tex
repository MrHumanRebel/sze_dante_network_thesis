\pagenumbering{roman}
\setcounter{page}{1}

\selecthungarian

%----------------------------------------------------------------------------
% Kivonat Magyarul 
%----------------------------------------------------------------------------
\chapter*{Kivonat}
% TODO: Távolítsd el a megjegyzést, ha mégis szeretnéd, hogy bekerüljön a tartalomjegyzékbe
%\addcontentsline{toc}{chapter}{Kivonat}

Szakdolgozatom célja egy élőzenei produkció hangrendszerének megtervezése, 
kiépítése, optimalizálása és beüzemelése, mely a Dante protokollra 
építkezik. Az Audio over IP rendszerek, különösen a Dante hálózatok, 
lehetőséget nyújtanak a hagyományos analóg hangrendszerekhez 
képes gyorsabb, megbízhatóbb és skálázhatóbb megoldások alkalmazására. 
A Dante protokoll technológiai előnyei közé tartozik az alacsony 
késleltetés, a magas szintű jelminőség és a hálózati redundancia 
biztosítása, ami különösen fontos az élő produkciók során.

Az 1. fejezet bevezeti a szakdolgozat témáját és eredetét,
miért is ezt a témát választotta a szerző.

A 2. fejezet az alapvető hangtechnikai fogalmakat és rendszereket 
mutatja be, beleértve az analóg és digitális hangrendszerek 
közötti eltéréseket. Ismerteti a digitális audiojelek 
feldolgozásának alapjait,kitér a hangerősségre, a hangnyomásszintre, valamint a 
különböző hangforrások, mint a pontsugárzók és a LineArray 
rendszerek közötti eltérésekre. 

A 3. fejezet célja, hogy alaposan bemutassa a digitális audió 
rendszerek alapjait és a Dante protokoll technikai részleteit. 
Ebben a részben többek között bemutatásra kerülnek az IP-címek kiosztásának módszerei, 
a hálózati topológiák kialakítása, 
valamint a különböző hálózati eszközök, mint például switchek szerepei.

A 4. fejezet a tervezési és telepítési folyamatot tárgyalja, 
különös figyelmet szentelve a felhasznált eszközöknek, 
mint például a hangprocesszorok, végerősítők és 
hangszórók. A tervezési szakasz kulcsfontosságú eleme a
rendszer méretezése és konfigurálása, amely figyelembe 
veszi a produkció követelményeit, mint például a helyszín 
méretét, a közönség elrendezését és az előadás 
specifikus igényeit. Ezen felül bemutatom a Dante 
Controller szoftver használatát, amely lehetővé teszi a 
hálózati eszközök konfigurálását és a forgalom 
monitorozását. 
A telepítés során a Dante-alapú hálózat  kiépítése mellett kiemelten
foglalkozom a hangminőség optimalizálásával. A valós környezetben 
végzett tesztelés során a rendszer megbízhatóságát és 
rugalmasságát is értékelem.

Az 5. fejezet a telepítés utáni üzemeltetési tapasztalatokkal 
foglalkozik. Részletesen bemutatom a rendszer integrálását a 
TéDé Rendezvények egy élőzenei produkcióján, ahol 
valós időben teszteltem a Dante hálózat teljesítményét. 
Itt tárgyalom a hangrendszer és a vezérlőrendszerek 
integrációjának fontosságát, valamint az előadás közben 
tapasztalt akusztikai és hálózati kihívások kezelését. 
Az eredmények alapján további fejlesztési lehetőségeket 
javaslok, melyek még tovább növelhetik a rendszer teljesítményét és rugalmasságát.

Az általam bemutatott rendszer lényeges előnyöket nyújt egy analóg 
hangrendszerekkel szemben.
Az üzemeltetési tapasztalatok azt igazolják, hogy 
az Audio over IP technológiák, különösen a Dante, 
stabil, skálázható és megbízható megoldást 
kínál a modern élőzenei produkciók számára.


\vfill
\selectenglish


%----------------------------------------------------------------------------
% Abstract in English
%----------------------------------------------------------------------------
\chapter*{Abstract}
% TODO: Távolítsd el a megjegyzést, ha mégis szeretnéd, hogy bekerüljön a tartalomjegyzékbe
%\addcontentsline{toc}{chapter}{Abstract}

The aim of my thesis is to design, build, optimize, and 
commission an audio system for a live music production, 
which is based on the Dante protocol. Audio over IP 
systems, particularly Dante networks, provide faster, 
more reliable, and scalable solutions compared to 
traditional analog sound systems. The technological 
advantages of the Dante protocol include low latency, 
high signal quality, and network redundancy, which are 
particularly important during live productions.

Chapter 1 introduces the topic of the thesis and its 
origin, explaining why the author chose this subject.

Chapter 2 presents fundamental audio engineering 
concepts and systems, highlighting the differences 
between analog and digital sound systems. It discusses 
the basics of processing digital audio signals, 
addresses volume levels, sound pressure levels, and 
differences between various sound sources, such as 
point sources and Line Array systems.

Chapter 3 aims to thoroughly present the foundations 
of digital audio systems and the technical details of 
the Dante protocol. This section includes a discussion 
on methods for assigning IP addresses, the design of 
network topologies, and the roles of various network 
devices, such as switches.

Chapter 4 covers the design and installation process, 
paying special attention to the utilized equipment, 
including audio processors, power amplifiers, and 
speakers. A crucial element of the design phase is 
the sizing and configuration of the system, which 
takes into account production requirements, such as 
the venue size, audience layout, and specific needs 
of the performance. Additionally, I present the use of 
the Dante Controller software, which allows for the 
configuration of network devices and traffic monitoring. 
During the installation, alongside building the Dante- 
based network, I focus on optimizing sound quality. 
In tests conducted in real environments, I also assess 
the reliability and flexibility of the system.

Chapter 5 addresses the operational experiences post- 
installation. I provide a detailed account of the 
integration of the system into a live music production 
by TéDé Rendezvények, where I tested the performance 
of the Dante network in real-time. This section discusses 
the importance of the integration between the sound 
system and control systems, as well as the management 
of acoustic and network challenges encountered during 
the performance. Based on the results, I propose further 
development opportunities that could enhance the 
performance and flexibility of the system.

The system I presented offers significant advantages 
over analog sound systems. The operational experiences 
demonstrate that Audio over IP technologies, particularly 
Dante, provide a stable, scalable, and reliable solution 
for modern live music productions.





\vfill
\selectthesislanguage

\newcounter{romanPage}
\setcounter{romanPage}{\value{page}}
\stepcounter{romanPage}