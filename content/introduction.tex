%----------------------------------------------------------------------------
\chapter{\bevezetes}
%----------------------------------------------------------------------------

%----------------------------------------------------------------------------
\section{A kezdetek}
%----------------------------------------------------------------------------

Kisgyermek koromtól kezdve érdekelnek a hangtechnikához fűződő eszközök és azok elméleti-gyakorlati működése. Első élményeim egyike közé tartozik az, amikor
szüleim egy új fajta rádiólejátszót vásároltak otthonra, amelyen már nem csak a rádióadásokat lehetett hallgatni, hanem lejátszhatóak voltak kazetták is.
A készüléket akkoriban jobban tudtam kezelni gyermekként, mint a szüleim, egyértelmű volt már akkoriban is, a technika és a zene iránti érdeklődésem.
Ezek után általános iskolában a fizika tanárommal együtt kezdük el a sulirádió működtetését, amelynek a telepítési részében is részt vettem, mivel azelőtt
csak egyszerű csengők voltak felszerelve az épületben. Szükség volt hangsugárzókra, erősítőkre, mikrofonokra, és egyéb kiegészítőkre. A rádió működtetése
is az én feladatom lett két másik barátommal együtt, az iskolával kapcsolatos híreket és információkat mondtuk be rövid szünetekben, a hosszabb szünetekben pedig zenéket játszottunk.
Mindeközben zeneiskolába is beiratkoztam, ahol ütőhangszeresként tanultam egészen egyetemi tanulmányaim kezdetéig. A több mint tíz év alatt, sok új ismeretet és tapasztalatot szereztem,
amit a későbbiekben mint zenész, mint hangosító tudtam hasznosítani. Megismerkedtem a különböző zenei stílusok egyedi hangzásvilágával, ami a későbbiekben a hangosításban is nagy segítségemre volt.
Középiskolai tanulmányaim alatt kezdtem el komolyabban foglalkozni a hangtechinka világával komolyabb szinten. 
A fizika tanárommal ugyanis korábban nem csak a sulirádiót működtettük, hanem az összes sulibulit és rendezvényt a faluban mi szolgáltuk ki technikailag.
Ezért mivel már középiskolába jártam, a későbbiekben egy feltörekvő fiatalos és modern gondolkodású magánvállalkozáshoz ajánlott be engem, mivel a fiatal és motivált munkaerőt kerestek. (TéDé Rendezvények)
Már az első munkalehetőségnél éreztem, hogy ez egy nagyon jó lehetőség számomra, mindenképpen szeretnék ebben a szakmában dolgozni.
A cég fő profilja a hangrendszerek kiépítése és üzemeltetése volt, de későbbiekben már a fénytechnikával és színpadtechnikával is el kezdtünk foglalkozni. 
Ettől kezdve kezdtem el aktívan dolgozni a rendezvényiparban a hangrendszerek világában. Az évek során egyre több tapasztalatot
szereztem. Évente több mint száz rendezvényen tudtam folyamatosan fejlődni, rutint és ismerettséget szerezni a szakmában.

\subsection{Téma választás}

A hangrendszerek világa az elmúlt években nagy változáson ment keresztül. A digitális technika térhódítása a hangtechnikában is megjelent, és egyre több fajta megoldás jelent meg a piacon.
A digitális fejlesztésekből mi sem szerettünk volna kimaradni, hogy hangtechnikai apparátusunk korszerű és versenyképes maradjon.
Ekkor jött a fejlesztési ötlet, egy olyan rendszert tervezni, amely teljes mértékben digitális alapokra helyezi a jelenlegi hibrid megoldásunkat. A cégvezetőtől azt a feladatot kaptam,
hogy tervezzek egy olyan rendszert, amely megoldja a jelenlegi hibrid rendszerünk teljes digitális megoldásra való cseréjét. 
Az alapvető szempontok közé tartozott, hogy a rendszer legyen könnyen skálázható, és bővíthető, valamint a jelenlegi rendszert minden tekintetben felülmúlja.
Első lépésben a különböző Audio over IP protokollok közül kellett választanom,
amely a leginkább megfelel a rendszerünk igényeinek. Ehhez a piacon lévő protokollokat kellett megvizsgálnom, és a legjobb megoldást választani. 

