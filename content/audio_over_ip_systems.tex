%----------------------------------------------------------------------------
\chapter{\AudioOverIp}
%----------------------------------------------------------------------------
\section{Bevezetés az Audio over IP világába \cite{AHNERT2023} \cite{MCCARTHY2016}}
%----------------------------------------------------------------------------
Az 1990-es évek óta az informatika és a hálózatok robbanásszerű fejlődésével együtt a professzionális audio ipar is elkezdett változni.
A `pontról pontra' elvű (minden eszközt külön-külön kell fizikailag összekapcsolni kábellel amely digitális információt hordoz) digitális audió átvitel helyett, mint például
az AES/EBU vagy a MADI, az IP alapú rendszerek felé kezdett el elmozdulni. 
Mivel ezek az IP rendszerek csomagalapúak, hatalmas rugalmasságot és továbbfejlesztési lehetőségeket biztosítanak a hagyományos rendszerekkel szemben.
Ezek az előnyök mind hardveres, mind szoftveres szinten megjelennek, és lehetővé teszik a rendszerek könnyebb kezelhetőségét, valamint a hálózatok egyszerűbb kiépítését és karbantartását.
%----------------------------------------------------------------------------
\subsection{Előnyök}
%----------------------------------------------------------------------------
Nincsen szükség fizikai kábelekre a különböző végpontok között. Egyetlen egy CAT kábelre van szükségünk a rendszerbe kapcsoláshoz, majd egy 
szoftveres kezelőfelületen keresztül bármikor megváltoztathatóak a jelek útjai. (Ez alól kivételt képeznek a redundáns rendszerek amelyek két CAT kábelt igényelnek a redundancia biztosítása végett)
A megfelelően megtervezett rendszerünk modulárissá válik, és a hálózat bármely pontján könnyen bővíthető, illetve a hálózat bármely pontjáról elérhetővé válik a rendszer. 
Egyes eszközökre a gyártó akár olyan fundamentális frissítéseket is kiadhat, amely nagymértékben képes lehet az adott eszköz funkcionalitását javítani, 
vagy új funkciók hozzáadásával tovább növelni a komponens értékét.
%----------------------------------------------------------------------------
\subsection{Hátrányok}
%----------------------------------------------------------------------------
A digitális hangfeldolgozás nagy flexibilitást és helytakarékosságot hozott el
magával ebben az iparágban, de mindez nem teljes mértékben jött hátrány nélkül.
Még a legmodernebb és legjobb A-D (analog to digital) konverterek sem tudták
teljes mértékben visszaadni azt a tipikus analog hangot, amire egy analog
hangrendszer képes. Ez elsősorban abból fakad, hogy egy átlagos digitális
keverővel és végfokrendszerrel ellátott új generációs hangrendszerben sok A-D és
D-A konverzió történik, és minden egyes konverzi-óval, hiába veszünk sok mintát
(96 kHz 24 bit 3000+kbps), a hang akkor is veszt egy kicsit a
karakterisztikájából.
%----------------------------------------------------------------------------
\subsection{Fázishelyesség}
%----------------------------------------------------------------------------

%----------------------------------------------------------------------------
\subsection{Időszinkronizáció}
%----------------------------------------------------------------------------

%----------------------------------------------------------------------------
\subsection{Mintavételi frekvencia és bitmélység}
%----------------------------------------------------------------------------

%----------------------------------------------------------------------------
\subsection{Késleltetés és bufferek}
%----------------------------------------------------------------------------

%----------------------------------------------------------------------------
\subsection{IP címek és maszkok}
%----------------------------------------------------------------------------

%----------------------------------------------------------------------------
\subsection{Unicast és Multicast}
%----------------------------------------------------------------------------

%----------------------------------------------------------------------------
\subsection{Redundancia}
%----------------------------------------------------------------------------

 %----------------------------------------------------------------------------
\section{AES67}
%----------------------------------------------------------------------------

%----------------------------------------------------------------------------
\section{Waves SoundGrid} 
%----------------------------------------------------------------------------
% https://www.soundonsound.com/reviews/wavesdigico-digigrid
% https://www.waves.com/support/soundgrid-system-design-guidelines
% https://www.fullcompass.com/common/files/12568-WhitePaper.pdf


%----------------------------------------------------------------------------
\section{Audinate Dante}
%----------------------------------------------------------------------------


