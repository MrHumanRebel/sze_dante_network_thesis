% !TeX root = ./thesis.tex
% !TeX spellcheck = hu_HU
% !TeX encoding = UTF-8
% !TeX program = pdflatex
%TODO Change language to en_GB (recommended) or en_US for English documents
\documentclass[11pt,a4paper,oneside]{report}             % Egyoldalas (javasolt)
%\documentclass[11pt,a4paper,twoside,openright]{report}  % Duplex

\input{include/packages}

\usepackage{float}



%TODO Saját adataiddal töltsd ki a kommentek szerint
%--------------------------------------------------------------------------------------
\newcommand{\szerzoVezeteknev}{Székely}
\newcommand{\szerzoKeresztnev}{Dániel}
\newcommand{\szerzoNeptun}{JAXC3C}

\newcommand{\szakirany}{} % Informatikusoknál nincs szakirány. Villamosmérnököknél: Automatizálás (\aut) vagy Infokommunikáció (\infokom).

\newcommand{\konzulensAMegszolitas}{}
\newcommand{\konzulensAVezeteknev}{Paál}
\newcommand{\konzulensAKeresztnev}{Dávid}
\newcommand{\konzulensBMegszolitas}{}
\newcommand{\konzulensBVezeteknev}{Tamás}
\newcommand{\konzulensBKeresztnev}{Dávid}
\newcommand{\konzulensCMegszolitas}{}
\newcommand{\konzulensCVezeteknev}{}
\newcommand{\konzulensCKeresztnev}{}

\newcommand{\cim}{Digitális audio - Dante protokollra épülő hangrendszer
tervezése, építése, optimalizálása, beüzemelése} % Cím
\newcommand{\tanszek}{\szeit} % informatika (\szeit), automatizálási (\szeaut) vagy távközlési (\szetat)
\newcommand{\doktipus}{\szakdolgozat} % Dokumentum típusa (\szakdolgozat, \diplomaterv vagy \dolgozat)
\newcommand{\szak}{\infoBSc} % Mérnökinformatikus BSc (\infoMsc), MSc (\infoMsc), Gazdaságinformatikus BSc (\gazdInfoBsc), MSc (\gazdInfoMsc), vagy Villamosmérnöki BSc (\villBSc), MSc (\villMSc)

\include{include/variables}
%TODO Nyelv beállítása
% Beállítások magyar nyelvű dolgozathoz
\input{include/thesis-hu}
% Settings for English documents
%\input{include/thesis-en}


\newcommand{\szerzoMeta}{\szerzoVezeteknev{} \szerzoKeresztnev} % egy szerző esetén TODO@FMA két szerző
\input{include/preamble} % beállítások, nem kell vele foglalkoznod remélhetőleg, de ha valami latex hekkelésre vagy új parancsra van szükséged annak itt a helye


%--------------------------------------------------------------------------------------
% Itt kezdődik a dolgozat
%--------------------------------------------------------------------------------------
\begin{document}

\selectthesislanguage

% Külső borító, minta kötéshez - csak elektronikus leadás esetén eltávolítandó
%~~~~~~~~~~~~~~~~~~~~~~~~~~~~~~~~~~~~~~~~~~~~~~~~~~~~~~~~~~~~~~~~~~~~~~~~~~~~~~~~~~~~~~
\include{include/outercover}

% Címoldal 
%~~~~~~~~~~~~~~~~~~~~~~~~~~~~~~~~~~~~~~~~~~~~~~~~~~~~~~~~~~~~~~~~~~~~~~~~~~~~~~~~~~~~~~
\include{include/titlepage}

%TODO Feladatkiíró lap helye, csak a nyomtatott verzijóba kerül az eredeti példány
%~~~~~~~~~~~~~~~~~~~~~~~~~~~~~~~~~~~~~~~~~~~~~~~~~~~~~~~~~~~~~~~~~~~~~~~~~~~~~~~~~~~~~~
\include{include/project}

% Nyilatkozat és Kivonat
%~~~~~~~~~~~~~~~~~~~~~~~~~~~~~~~~~~~~~~~~~~~~~~~~~~~~~~~~~~~~~~~~~~~~~~~~~~~~~~~~~~~~~~
\include{include/declaration} % ez legenerálódik magától a fentebb megadott adatok alapján
\pagenumbering{roman}
\setcounter{page}{1}

\selecthungarian

%----------------------------------------------------------------------------
% Kivonat Magyarul 
%----------------------------------------------------------------------------
\chapter*{Kivonat}
% TODO: Távolítsd el a megjegyzést, ha mégis szeretnéd, hogy bekerüljön a tartalomjegyzékbe
%\addcontentsline{toc}{chapter}{Kivonat}

Szakdolgozatom célja egy élőzenei produkció hangrendszerének megtervezése, 
kiépítése, optimalizálása és beüzemelése, mely a Dante protokollra 
építkezik. Az Audio over IP rendszerek, különösen a Dante hálózatok, 
lehetőséget nyújtanak a hagyományos analóg hangrendszerekhez 
képes gyorsabb, megbízhatóbb és skálázhatóbb megoldások alkalmazására. 
A Dante protokoll technológiai előnyei közé tartozik az alacsony 
késleltetés, a magas szintű jelminőség és a hálózati redundancia 
biztosítása, ami különösen fontos az élő produkciók során.

Az 1. fejezet bevezeti a szakdolgozat témáját és eredetét,
miért is ezt a témát választotta a szerző.

A 2. fejezet az alapvető hangtechnikai fogalmakat és rendszereket 
mutatja be, beleértve az analóg és digitális hangrendszerek 
közötti eltéréseket. Ismerteti a digitális audiojelek 
feldolgozásának alapjait,kitér a hangerősségre, a hangnyomásszintre, valamint a 
különböző hangforrások, mint a pontsugárzók és a LineArray 
rendszerek közötti eltérésekre. 

A 3. fejezet célja, hogy alaposan bemutassa a digitális audió 
rendszerek alapjait és a Dante protokoll technikai részleteit. 
Ebben a részben többek között bemutatásra kerülnek az IP-címek kiosztásának módszerei, 
a hálózati topológiák kialakítása, 
valamint a különböző hálózati eszközök, mint például switchek szerepei.

A 4. fejezet a tervezési és telepítési folyamatot tárgyalja, 
különös figyelmet szentelve a felhasznált eszközöknek, 
mint például a hangprocesszorok, végerősítők és 
hangszórók. A tervezési szakasz kulcsfontosságú eleme a
rendszer méretezése és konfigurálása, amely figyelembe 
veszi a produkció követelményeit, mint például a helyszín 
méretét, a közönség elrendezését és az előadás 
specifikus igényeit. Ezen felül bemutatom a Dante 
Controller szoftver használatát, amely lehetővé teszi a 
hálózati eszközök konfigurálását és a forgalom 
monitorozását. 
A telepítés során a Dante-alapú hálózat  kiépítése mellett kiemelten
foglalkozom a hangminőség optimalizálásával. A valós környezetben 
végzett tesztelés során a rendszer megbízhatóságát és 
rugalmasságát is értékelem.

Az 5. fejezet a telepítés utáni üzemeltetési tapasztalatokkal 
foglalkozik. Részletesen bemutatom a rendszer integrálását a 
TéDé Rendezvények egy élőzenei produkcióján, ahol 
valós időben teszteltem a Dante hálózat teljesítményét. 
Itt tárgyalom a hangrendszer és a vezérlőrendszerek 
integrációjának fontosságát, valamint az előadás közben 
tapasztalt akusztikai és hálózati kihívások kezelését. 
Az eredmények alapján további fejlesztési lehetőségeket 
javaslok, melyek még tovább növelhetik a rendszer teljesítményét és rugalmasságát.

Az általam bemutatott rendszer lényeges előnyöket nyújt egy analóg 
hangrendszerekkel szemben.
Az üzemeltetési tapasztalatok azt igazolják, hogy 
az Audio over IP technológiák, különösen a Dante, 
stabil, skálázható és megbízható megoldást 
kínál a modern élőzenei produkciók számára.


\vfill
\selectenglish


%----------------------------------------------------------------------------
% Abstract in English
%----------------------------------------------------------------------------
\chapter*{Abstract}
% TODO: Távolítsd el a megjegyzést, ha mégis szeretnéd, hogy bekerüljön a tartalomjegyzékbe
%\addcontentsline{toc}{chapter}{Abstract}

The aim of my thesis is to design, build, optimize, and 
commission an audio system for a live music production, 
which is based on the Dante protocol. Audio over IP 
systems, particularly Dante networks, provide faster, 
more reliable, and scalable solutions compared to 
traditional analog sound systems. The technological 
advantages of the Dante protocol include low latency, 
high signal quality, and network redundancy, which are 
particularly important during live productions.

Chapter 1 introduces the topic of the thesis and its 
origin, explaining why the author chose this subject.

Chapter 2 presents fundamental audio engineering 
concepts and systems, highlighting the differences 
between analog and digital sound systems. It discusses 
the basics of processing digital audio signals, 
addresses volume levels, sound pressure levels, and 
differences between various sound sources, such as 
point sources and Line Array systems.

Chapter 3 aims to thoroughly present the foundations 
of digital audio systems and the technical details of 
the Dante protocol. This section includes a discussion 
on methods for assigning IP addresses, the design of 
network topologies, and the roles of various network 
devices, such as switches.

Chapter 4 covers the design and installation process, 
paying special attention to the utilized equipment, 
including audio processors, power amplifiers, and 
speakers. A crucial element of the design phase is 
the sizing and configuration of the system, which 
takes into account production requirements, such as 
the venue size, audience layout, and specific needs 
of the performance. Additionally, I present the use of 
the Dante Controller software, which allows for the 
configuration of network devices and traffic monitoring. 
During the installation, alongside building the Dante- 
based network, I focus on optimizing sound quality. 
In tests conducted in real environments, I also assess 
the reliability and flexibility of the system.

Chapter 5 addresses the operational experiences post- 
installation. I provide a detailed account of the 
integration of the system into a live music production 
by TéDé Rendezvények, where I tested the performance 
of the Dante network in real-time. This section discusses 
the importance of the integration between the sound 
system and control systems, as well as the management 
of acoustic and network challenges encountered during 
the performance. Based on the results, I propose further 
development opportunities that could enhance the 
performance and flexibility of the system.

The system I presented offers significant advantages 
over analog sound systems. The operational experiences 
demonstrate that Audio over IP technologies, particularly 
Dante, provide a stable, scalable, and reliable solution 
for modern live music productions.





\vfill
\selectthesislanguage

\newcounter{romanPage}
\setcounter{romanPage}{\value{page}}
\stepcounter{romanPage} %TODO ezt át kell írnod

% Tartalomjegyzék
%~~~~~~~~~~~~~~~~~~~~~~~~~~~~~~~~~~~~~~~~~~~~~~~~~~~~~~~~~~~~~~~~~~~~~~~~~~~~~~~~~~~~~~
\tableofcontents\vfill

% A dolgozat lényegi része
%~~~~~~~~~~~~~~~~~~~~~~~~~~~~~~~~~~~~~~~~~~~~~~~~~~~~~~~~~~~~~~~~~~~~~~~~~~~~~~~~~~~~~~
\pagenumbering{arabic}

%TODO készítsd el a saját munkád
%----------------------------------------------------------------------------
\chapter{\bevezetes}
%----------------------------------------------------------------------------

%----------------------------------------------------------------------------
\section{A kezdetek}
%----------------------------------------------------------------------------
Kisgyermek koromtól kezdve érdekelnek a hangtechnikához fűződő eszközök és azok elméleti-gyakorlati működése. Első élményeim egyike közé tartozik az, amikor
szüleim egy új fajta rádiólejátszót vásároltak otthonra, amelyen már nem csak a rádióadásokat lehetett hallgatni, hanem lejátszhatóak voltak kazetták is.
A készüléket akkoriban jobban tudtam kezelni gyermekként, mint a szüleim, egyértelmű volt már akkoriban is, a technika és a zene iránti érdeklődésem. 

Ezek után általános iskolában a fizika tanárommal együtt kezdtük el a sulirádió működtetését, amelynek a telepítési részében is részt vettem.
Korábban az egyszerű klasszikus csengők voltak felszerelve az épületben. Szükség volt hangsugárzókra, erősítőkre, mikrofonokra, és egyéb kiegészítőkre. A rádió működtetése
is az én feladatom lett két barátommal együtt, az iskolával kapcsolatos híreket és információkat mondtuk be röviden a szünetekben, hosszabb szünetekben pedig zenéket is játszottunk.

Mindeközben zeneiskolába is beiratkoztam, ahol ütőhangszeresként tanultam egészen egyetemi tanulmányaim kezdetéig. A több mint tíz év alatt, sok új ismeretet és tapasztalatot szereztem,
amit a későbbiekben mint zenész és hangtechnikával aktívan foglalkozó szakember tudtam hasznosítani. Megismerkedtem a különböző zenei stílusok egyedi hangzásvilágával, ami a későbbiekben a hangosításban is nagy segítségemre volt.

Középiskolai tanulmányaim alatt kezdtem el komolyabban foglalkozni a hangtechnika világával.
Előbbiekben említett tanárommal ugyanis korábban nem csak a sulirádiót működtettük, hanem az összes sulibulit és helyi rendezvényt mi szolgáltuk ki technikailag.
Mikor már középiskolába jártam, egy feltörekvő fiatalos és modern gondolkodású magánvállalkozáshoz ajánlottak be engem, ahol fiatal és motivált munkaerőt kerestek.

Már az első munkalehetőségnél éreztem, hogy ez egy nagyon jó lehetőség lehet a számomra, mindenképpen szeretnék ebben a szakmában tevékenykedni.
A cég fő profilja a hangrendszerek rendezvényekre való kiépítése és üzemeltetése volt, de későbbiekben bővült a portfólió és már fénytechnikával és színpadtechnikával is elkezdett foglalkozni. 
Ettől a ponttól kezdve kezdtem el aktívan dolgozni a rendezvényiparban és a hangrendszerek világában. Az évek során egyre több tapasztalatot szereztem.
Évente több mint száz rendezvényen tudtam folyamatosan fejlődni, rutint és ismeretséget szerezni a szakmában.
%----------------------------------------------------------------------------
\section{Témaválasztás}
%----------------------------------------------------------------------------0
A hangrendszerek világa az elmúlt évtizedekben nagy változásokon ment keresztül.
A digitális technika térhódítása a hangtechnikában is megjelent, és egyre több fajta új megközelítés jelent meg a piacon.
Ezekből a fejlesztésekből mi sem szerettünk volna kimaradni, hogy hangtechnikai apparátusunk korszerű és versenyképes maradjon.

Ekkor jött a fejlesztési ötlet, egy olyan rendszert tervezni, amely teljes mértékben digitális alapokra helyezi a jelenlegi hibrid megoldásunkat.
A cégvezetőtől azt a feladatot kaptam mint leendő informatikus mérnök, hogy tervezzek egy olyan rendszert,
amely megoldja a jelenlegi hibrid rendszerünk teljes digitális megoldásra való átállását.
Az alapvető szempontok közé tartozott, hogy a rendszer legyen könnyen skálázható, bővíthető, valamint a jelenlegi rendszert minden tekintetben múlja felül.

Különböző Audio over IP protokollok léteznek, ezért választanom kellett, amely a leginkább megfelel az aktuális igényeinek és anyagi helyzetünknek.
Ehhez a piacon lévő protokollokat kellett megvizsgálnom, és egy optimális megoldást választani.

Szakdolgozatomban sok idegen nyelvű szóra nincsen megfelelő és pontos fordítás ami teljes mértékben tükrözné az adott kifejezés jelentését.
A továbbiakban ebből kifolyólag feltételezve azt, hogy az olvasó tisztában van a szakmai terminológiával angolul fogom említeni a szakmai kifejezéseket.
Ezek a kifejezések megjelenhetnek írt szövegként és ábrás illusztrációkban is. Egyes ábrák rendkívüli komplexitásuk végett a szakmai helyesség és precízió érdekében
külső forrásból származnak, ezeket az ábrákat a forrásukkal együtt fogom megjeleníteni.

%----------------------------------------------------------------------------
\chapter{\AudioOverIp}
%----------------------------------------------------------------------------
% https://broadmax.hu/wp-content/uploads/2016/10/three-pillars-audio-networking-whitepaper-audinate.pdf
% https://broadmax.hu/wp-content/uploads/2016/10/rise-of-audio-networking-audinate-wp.pdf
% https://broadmax.hu/blog/audio-over-ip/
% https://rhconsulting.uk/blog/networked-audio-products-2023/
% Sound Reinforcement for Audio Engineers from page 308 to 346

%----------------------------------------------------------------------------
\section{Bevezetés az Audio over IP világába} \cite{AHNERT2023} \cite{MCCARTHY2016}
%----------------------------------------------------------------------------

Az 1990-es évek óta az informatika és a hálózatok robbanásszerű fejlődésével együtt a professzionális audio ipar is elkezdett változni.
A `pontról pontra' elvű (minden eszközt külön-külön kell fizikailag összekapcsolni kábellel amely digitális információt hordoz) digitális audió átvitel helyett, mint például
az AES/EBU vagy a MADI, az IP alapú rendszerek felé kezdett el elmozdulni. 
Mivel ezek az IP rendszerek csomagalapúak, hatalmas rugalmasságot és továbbfejlesztési lehetőségeket biztosítanak a hagyományos rendszerekkel szemben.
Ezek az előnyök mind hardveres, mind szoftveres szinten megjelennek, és lehetővé teszik a rendszerek könnyebb kezelhetőségét, valamint a hálózatok egyszerűbb kiépítését és karbantartását.

%----------------------------------------------------------------------------
\subsection{Előnyök}
%----------------------------------------------------------------------------

Nincsen szükség fizikai kábelekre a különböző végpontok között. Egyetlen egy CAT kábelre van szükségünk a rendszerbe kapcsoláshoz, majd egy 
szoftveres kezelőfelületen keresztül bármikor megváltoztathatóak a jelek útjai. (Ez alól kivételt képeznek a redundáns rendszerek amelyek két CAT kábelt igényelnek a redundancia biztosítása végett)
A megfelelően megtervezett rendszerünk modulárissá válik, és a hálózat bármely pontján könnyen bővíthető, illetve a hálózat bármely pontjáról elérhetővé válik a rendszer. 
Egyes eszközökre a gyártó akár olyan fundamentális frissítéseket is kiadhat, amely nagymértékben képes lehet az adott eszköz funkcionalitását javítani, 
vagy új funkciók hozzáadásával tovább növelni a komponens értékét.

%----------------------------------------------------------------------------
\subsection{Hátrányok}
%----------------------------------------------------------------------------

A digitális hangfeldolgozás nagy flexibilitást és helytakarékosságot hozott el
magával ebben az iparágban, de mindez nem teljes mértékben jött hátrány nélkül.
Még a legmodernebb és legjobb A-D (analog to digital) konverterek sem tudták
teljes mértékben visszaadni azt a tipikus analog hangot, amire egy analog
hangrendszer képes. Ez elsősorban abból fakad, hogy egy átlagos digitális
keverővel és végfokrendszerrel ellátott új generációs hangrendszerben sok A-D és
D-A konverzió történik, és minden egyes konverzi-óval, hiába veszünk sok mintát
(96 kHz 24 bit 3000+kbps), a hang akkor is veszt egy kicsit a
karakterisztikájából.


%----------------------------------------------------------------------------
\subsection{Fázishelyesség}
%----------------------------------------------------------------------------


%----------------------------------------------------------------------------
\subsection{Időszinkronizáció}
%----------------------------------------------------------------------------


%----------------------------------------------------------------------------
\subsection{Mintavételi frekvencia és bitmélység}
%----------------------------------------------------------------------------

%----------------------------------------------------------------------------
\subsection{Késleltetés és bufferek}
%----------------------------------------------------------------------------

%----------------------------------------------------------------------------
\subsection{IP címek és maszkok}
%----------------------------------------------------------------------------

%----------------------------------------------------------------------------
\subsection{Unicast és Multicast}
%----------------------------------------------------------------------------


%----------------------------------------------------------------------------
\subsection{Redundancia}
%----------------------------------------------------------------------------

 

%----------------------------------------------------------------------------
\section{AES67}
%----------------------------------------------------------------------------


%----------------------------------------------------------------------------
\section{Waves SoundGrid} 
%----------------------------------------------------------------------------
% https://www.soundonsound.com/reviews/wavesdigico-digigrid
% https://www.waves.com/support/soundgrid-system-design-guidelines
% https://www.fullcompass.com/common/files/12568-WhitePaper.pdf


%----------------------------------------------------------------------------
\section{Audinate Dante}
%----------------------------------------------------------------------------



%----------------------------------------------------------------------------
\chapter{\SystemDesign}
%----------------------------------------------------------------------------
\section{Követelmények}
%----------------------------------------------------------------------------
Az új rendszer tervezésekor a következő szempontokat kellett szem előtt tartanom:
\begin{itemize}
	\item Teljesen digitális megoldás kialakítása.
	\item Redundáns rendszer kialakítása biztosítva a folyamatos működést.
	\item Rugalmas felépítés, amely lehetővé teszi a könnyű alkalmazkodást változó körülményekhez.
	\item Könnyen bővíthető struktúra kialakítása a jövőbeli igényekhez való gyors reagálás érdekében.
	\item Magasabb hangminőség és hangnyomás elérése a korábbi rendszerrel összehasonlítva.
	\item Jobb lefedettség és egyenletes hangvisszaadás biztosítása a közönség területén.
	\item A piacon lévő termékekhez képest viszonylag költséghatékony megoldás kidolgozása.
	\item Legyen hosszú távon egy kompetitív és korszerű rendszer.
	\item Megbízható és kiforrott technológiára épüljön.
\end{itemize}
%----------------------------------------------------------------------------
\section{Rendszerterv és kivitelezés}
%----------------------------------------------------------------------------
A cég alapítása óta Martin Audio termékeket használ, a régi W8LM rendszerrel nagyon elégedettek voltunk, 
ebből kifolyólag a választás márka szempontjából nem is volt kérdéses. A Martin új rendszerei a Wavefront Precision sorozat szériás
végfokai pedig a Dante hálózatot használja a digitális hangátvitelhez.
Ebből kifolyólag a választás a Dante protokollra esett, így a szakdolgozatomban 
megtervezésre kerülő hangrendszer lelke ez a protokoll lesz.
Hangládák szempontjából a Martin Audio Wavefront Precision szériás termékein belül kétfajta rendszerre is esett választás.
Ezekről a későbbiekben részletesen lesz szó.
Mivel teljes mértékben digitális rendszert elérése volt a cél, ezért a Dante modullal
rendelkező Martin Audio iKON iK42 és iK81 végfokok tökéletesen illeszkedtek a rendszerbe.
Mindkét végfok csúcskategóriás teljesítményt és hangminőséget nyújt, és a D kategóriás
erősítőknek köszönhetően rendkívül kis helyet foglalnak el a rack-ben miközben kiváló hatásfok
mellett képesek nagy teljesítményt leadni. A két erősítő között két fundamentális különbség van,
az iK42 négy kimeneti csatornával rendelkezik, 20.000 W teljesítmény leadása mellett. 
Ezzel szemben az iK81 nyolc kimeneti csatornával rendelkezik, 10.000 W teljesítménnyel.~\cite{IKONAMPUSEGUIDE}
A végfokokat a Dante hálózaton keresztül fogjuk jellel ellátni, így a hagyományos analóg XLR, vagy AES kábelezés helyett
két CAT5E kábellel (a redundancia miatt) tudjuk a végfokokat a hálózatra kötni. 
%----------------------------------------------------------------------------
\begin{figure}[H]
    \centering
    \begin{minipage}{0.45\textwidth}
        \centering
        \includegraphics[width=\linewidth, keepaspectratio]{figures/ikon_ik42.jpg}
        \caption{Martin Audio iK42 végfok}\label{fig:ikon_ik42}
    \end{minipage}\hfill
    \begin{minipage}{0.45\textwidth}
        \centering
        \includegraphics[width=\linewidth, keepaspectratio]{figures/ikon_ik81.jpg}
        \caption{Martin Audio iK81 végfok}\label{fig:ikon_ik81}
    \end{minipage}
\end{figure}
%----------------------------------------------------------------------------
A végfokrendszer fő vezérlő protokollja miatt szükség lesz még egy CAT5E alapú
összeköttetésre, ami az egyes végfokokat köti össze egy hálózatba a switcheken keresztül. (VU-NET protokoll)
Minden egyes végfokrackben két MikroTik 24 portos switch lesz. 
Ezekre az eszközökre készítettem egy unified konfigurációt, amely minden olyan felesleges biztonsági
beállítást kikapcsol, amelyekre egy normál internetes hálózatban szükség lenne, de egy zárt hálózatban ahol kizárólag
a Dante eszközök kommunikálnak egymással, ezek a beállítások csak felesleges terhelést jelentenének a hálózaton.
Valamint amennyiben még több eszközt szeretnénk a hálózatra kötni, akkor a switchek gyorsan üzembe helyezhetőek,
mivel a konfiguráció már előre készen áll, csak fel kell tölteni a beállításokat, valamint firmware-t egyeztetni.
Az egyik switch a Dante elsődleges hálózatát fogja kizárólag kezelni.
A másik switch a Dante másodlagos hálózatát, és a VU-NET hálózatot fogja kezelni.
Ez a két alhálózat VLAN szegmensekbe lesz elkülönítve, hogy a hálózat biztosan stabil legyen, ne fordulhassanak elő csomagütközések.
A rendszer kábelezése egyedileg készített Neutrik EtherCon csatlakozóval ellátott CAT5E és CAT6A kábelekből fog állni.
A kábelek készítésekor a T568B szabvány szerinti kábelrendezést alkalmazom, mivel a T568B jobb átviteli
teljesítményt és interferencia védelmet nyújt a hosszú távú használat során.
Nem szükséges a T568A szabvány által nyújtott plusz kompatibilitás a régi hálózatokkal, mivel egy teljesen új 
modern hálózatot fogunk kiépíteni. 
A csatlakozók felhelyezésekor a szabványnak megfelelően járok el, milliméterre pontosan, hogy a kábelek
a lehető legjobb minőségűek legyenek. Ez a CAT6A kábeleknél különösen fontos, mivel a CAT6A kábelek
még nagyobb frekvencia tartományban képesek adatokat továbbítani, mint a CAT5E kábelek,
ezért jobban érzékenyek a külső interferenciákra.
A végfokokból egy egyedi patch panel segítségével vezetjük ki a végpontokat, hogy ne az eszköz saját csatlakozóját
degradáljuk a sokszori csatlakoztatás során, hanem a patch panelen lévőt, melyet amennyiben szükséges könnyen cserélhetünk.
Valamint az eszközön lévő csatlakozó egy szabványos RJ45 csatlakozó nem pedig egy Neutrik EtherCon csatlakozó.
Számunkra ez is fontos, mivel a Neutrik csatlakozók strapabíróbbak, és a kábelvégek is jobban védettek a sérülésektől,
és nehezen vagy egyáltalán nem lehet véletlenül kihúzni a csatlakozót a helyéről.
Az egyes végfokrackek összeköttetésért egy hibrid kábel lesz a felelős, amelyben négy darab CAT6A kábel van,
mindegyik véget külön színkóddal látjuk el, annak érdekében, hogy gyorsan és egyértelműen tudjuk azonosítani a kábeleket,
amikor egy adott helyszínen építjük össze a felszerelést, akár sok-sok óra munka után fáradtan.
A zöld szín jelzi majd a Dante elsődleges hálózatot, a kék szín a Dante másodlagos hálózatot, a piros szín a VU-NET hálózatot,
a fekete vég pedig egy tartalék kábel lesz, amennyiben valamelyik kábel megsérülne, vagy egyéb okból nem működne.
Mivel az előbbiekben már említett 24 portos switcheket használjuk, ezért ha vendég vagy bérelt eszközöket szeretnénk a hálózatra kötni,
akkor a switchekben található portokból ki tudjuk választani a megfelelő VLAN szegmenst, és a hozzá tartozó portokat,
így szabványos RJ45 csatlakozóval ellátott kábeleket Ethercon nélkül is tudunk használni.
Mindegyik rack hátuljában található még egy Wi-Fi router is, amely általában a VU-NET hálózatra kapcsolódik,
így a teljes rendszer vezérlése Wi-Fi-n keresztül is lehetséges lesz.
A Dante patch-et általában vezetékes módon fogjuk elkészíteni, de amennyiben készülékünk nem rendelkezik RJ45 csatlakozóval,
ami sajnos manapság egyre gyakoribb, akkor a Wi-Fi router segítségével is tudjuk konfigurálni a shared mode bekapcsolásával 
a Dante vezérlőben.
Az FOH (Front of House) pultot szintén egy hibrid kábel fogja összekötni a végfokrackekkel, amelyben viszont
már csak két darab CAT5E, két darab DMX és egy darab 230V-os tápkábel lesz. Ezzel a megoldással időt és helyet spórolunk,
az építésnél, mivel nem kell külön-külön vezetékeket kihúzni, elég egy kábelt, amelyben minden szükséges vezeték benne van.
Tehát összefoglalva három egymástól teljesen elkülönített hálózatot fogunk kiépíteni a rendszer kiszolgálására,
a Dante elsődleges és másodlagos hálózatot, valamint a VU-NET hálózatot.
%----------------------------------------------------------------------------
\begin{figure}[H]
	\centering
	\includegraphics[width=\linewidth, keepaspectratio]{figures/live-analog.jpg}
	\caption{Példa hagyományos analóg kábelezésre~\cite{APPLICATIONDIAGRAMSFORDANTESYSTEMS}}\label{fig:live-analog}
\end{figure}
%----------------------------------------------------------------------------
\begin{figure}[H]
	\centering
	\includegraphics[width=\linewidth, keepaspectratio]{figures/live-dante.jpg}
	\caption{Példa Dante digitális kábelezésre~\cite{APPLICATIONDIAGRAMSFORDANTESYSTEMS}}\label{fig:live-dante}
\end{figure}
%----------------------------------------------------------------------------
\subsection{Martin Audio Wavefront Precision hangrendszer}
%----------------------------------------------------------------------------
\subsubsection{Martin Audio Display 2.3.4 b1 tervező szoftver~\cite{DISPLAY23USERGUIDE}}
%----------------------------------------------------------------------------
Mielőtt bele fognánk a tervezési folyamatba, fontos megemlíteni, hogy a szoftver
eredetileg Intel alapú processzorokra lett tervezve és MatLab alapú. Ebből fakadóan
AMD Ryzen processzorokon habár elindult a szoftver, de nem volt stabil és a számítások során
minden esetben összeomlott, és használhatatlanul lassú volt. Személy szerint a saját gépem amivel dolgoztam
sajnos ilyen processzorral van szerelve ezért muszáj volt megoldást találni a problémára.
A Martin Audio hivatalos szoftveres támogatásához fordultam először, de sajnos nem tudtak segíteni.
Ezért a szoftver használatához
sok belefektetett óra olvasás után sikerült egy olyan MatLab CMD parancsot találnom, amivel
a szoftver elindul és használható.
Miután rájöttem a probléma gyökerére, ezt megosztottam velük, hogy a jövőben másoknak ne kelljen
ezzel a problémával szembesülniük.
A hiba az alábbi volt. Az új AMD Ryzen processzorok másfajta utasításkészletet használnak.
Ebből kifolyólag a MatLab 2015-s runtime alapú szoftver adta alaputasításokat nem tudta értelmezni a CPU.
A vezető szoftvermérnökkel való e-mail-es beszélgetésünk során megköszönte a probléma
megoldását, és nemsokkal a megoldásom megosztása után a hivatalos oldalra is felekerült
az indító parancsfájl. Az e-mailben további kollaborációra is adott lehetőséget.
A kompatibilitási problémát rögtön a script elején megoldottam,
mivel a következő parancs megadásával már használhatóvá válik a program: \texttt{set MKL\_DEBUG\_CPU\_TYPE=5} \newline
Ez a sor a program vezérlését AVX2-re állítja át, és mivel ezt az utasításkészletet már ismeri az AMD Ryzen processzor
is ezért a probléma már a múlté.
Az indító fájl további sorai optimalizálások a számítások gyorsítására, és a párhuzamosítására, ezzel jobban kihasználva
a rendelkezésre álló hardver erőforrásokat.\newpage
%----------------------------------------------------------------------------
\begin{lstlisting}[caption={A Display 2.3.4 b1 indító ".bat" scriptje AMD Ryzen processzorokhoz}, label=batcode, xleftmargin=\parindent]
    @echo off
    set PATH=%PATH%;C:\Program Files\Martin Audio\Display2_3_4_b1\application
    set MKL_DEBUG_CPU_TYPE=5
    set options=optimoptions('ga','UseParallel',true,'UseVectorized',false)
    set options=optimoptions('gamultiobj','UseParallel',true,'UseVectorized',false)
    set options=optimoptions('paretosearch','UseParallel',true)
    set options=optimoptions('particleswarm','UseParallel',true,'UseVectorized',false)
    set options=optimoptions('patternsearch','UseParallel',true,'UseCompletePoll',true,'UseVectorized',false)
    set options=optimoptions('surrogateopt','UseParallel',true)
    set GPUAcceleration=on
    start "Martin Audio" Display2_3_4_b1.exe
    pause
\end{lstlisting}
%----------------------------------------------------------------------------
\begin{figure}[H]
	\centering
	\includegraphics[width=\textwidth, keepaspectratio]{figures/ambrose_email.png}
	\caption{E-mail a Martin Audio vezető szoftvermérnökétől}
	\label{fig:ambrose_email}
\end{figure}
%----------------------------------------------------------------------------
Most, hogy már a szoftver használható és teljes mértékben működőképes, kezdjük el a tervezést.
A modellezés során a budapesti Millenáris B csarnoka lesz a referencia helyszín. Két LineArray rendszert fogunk
tervezni, mivel a terem hosszúsága és a lefedettség növelése miatt szükségünk lesz Delay kiegészítésre a fő hangrendszerhez.
Első lépésben a fő hangrendszert tervezem meg, ami oldalanként (bal és jobb) 8 darab WPC LineArray modulból fog állni.
Ez a láda 2 darab 10"-os mélysugárzót (LF), 2 darab 5"-os közép sugárzót (MF) és 4 darab 0.7"-os magassugárzót tartalmaz (HF).
Három utas Bi-amp meghajtású külső végfokot igénylő rendszer, ahol a mély tartományt (+1,-1) és a középmagas tartományt (+2,-2) külön kezeljük,
a négy pólusú Neutrik Speakon csatlakozókon keresztül.
A láda maximális hangnyomás szintje 135 dB, és 65 Hz-től 18 kHz-ig terjed a frekvencia átvitele +- 3 dB pontossággal.~\cite{WPCUSERGUIDE}
%----------------------------------------------------------------------------
\begin{figure}[H]
	\centering
	\includegraphics[width=80mm, keepaspectratio]{figures/wpc_front_view.jpg}
	\caption{Martin Audio WPC LineArray modul}\label{fig:wpc}
\end{figure}
%----------------------------------------------------------------------------
A program megnyitásakor a legelső lépés, hogy kiválasztjuk a termékpalettából a megfelelő hangrendszert.
Jelen esetben az előbbiekben említett WPC-t. A produkció igényei, a nagy létszámú közönség és a ládamennyiség miatt a rendszert
\textit{``riggelni''} fogjuk. (maximálisan 6 darab WPC-t lehet \textit{``stackelni''}, azaz a földre vagy mélyládákra helyezni)
A helyszín felmérése után a hangrendszer \textit{``riggelése''} lehetséges, mivel a csarnokban található tartószerkezet biztonságosan
és tartósan képes elviselni a rendszer súlyát.
A telepítés módja kiválasztása után megadjuk a szoftvernek a tervezni való hangláda mennyiséget, ez az esetünkben már említett 8 darab.
A hozzáadás gombra kattintva a elénk kerül a fő kezelőfelület, ahol a hangrendszert tudjuk lépésről lépésre tervezni.
%----------------------------------------------------------------------------
\begin{figure}[H]
	\centering
	\includegraphics[width=40mm, keepaspectratio]{figures/display_wpc_0.png}
	\caption{Display 2.3.4 b1 kezdőképernyője (WPC)}\label{fig:display_wpc_0}
\end{figure}
%----------------------------------------------------------------------------
A tervezési folyamat öt részre osztható, amiket a szoftverben külön kezelünk.
Ezeket a \textit{``Slice''}, \textit{``Cover''}, \textit{``Splay''}, \textit{``Rig''} és \textit{``EQ''} kezelőfelületeken tudjuk elvégezni,
balról jobbra haladva. Mivel a különböző részegységek egymásra épülnek, ezért fontos a sorrend betartása.
(tervezés utáni módosításokra természetesen van lehetőség, de az adott projekt első tervezési folyamata során ezeket a lépéseket kell követni)
%----------------------------------------------------------------------------
\begin{figure}[H]
	\centering
	\includegraphics[width=\textwidth, keepaspectratio]{figures/display_wpc_0_1.png}
	\caption{Display 2.3.4 b1 fő kezelőfelülete (WPC)}\label{fig:display_wpc_0_1}
\end{figure}
%----------------------------------------------------------------------------
A \textit{``Slice''} panelen meghatározzuk a rendszer fizikai pozícióját térben. A csarnok
pontos lemodellezése érdekében a mérésekhez lézeres távolságmérőt használtam.
Mivel minden egyes rendezvényen más és más a különböző elemek elhelyezkedése, ezért a
rendszert minden alkalommal újra kell tervezni, még akkor is ha maga a helyszín nem változik.
\textit{``Vertex''} pontok segítségével tudjuk a méreteket és a pozíciókat meghatározni.
A 2D-s modellen figyelembe kell venni a terem önálló méretén kívül a színpadod és a színpad mögötti területet is.
A rajznak tartalmazni kell azokat a falfelületeket is amelyeknél a hangvisszaverődést minimalizálni szeretnénk,
ennek az optimalizáció későbbi fázisában lesz jelentősége.
A terem pontos rajza után még két fontos paramétert kell megadni ezen a felületen.
El kell helyeznünk magát a hangrendszert a teremben, és meg kell határoznunk milyen magasra szeretnénk a rendszert emelni.
Mivel a csarnok rendkívül hosszú, és a adottságai megengedik, ezért a rendszert minél magasabbra szeretnénk emelni,
a jobb lefedettség érdekében.
A másik fontos paraméter az optimalizációhoz, a közönség területének meghatározása. Kezdő és végpont segítségével
tudjuk a területet meghatározni, ahol a hallgatóközönség tartózkodni fog.
%----------------------------------------------------------------------------
\begin{figure}[H]
	\centering
	\includegraphics[width=\textwidth, keepaspectratio]{figures/display_wpc_1.png}
	\caption{Display 2.3.4 b1 \textit{``Slice''} kezelőfelülete (WPC)}\label{fig:display_wpc_1}
\end{figure}
%----------------------------------------------------------------------------
A következő lépések a \textit{``Cover''} kezelőfelületen történnek.
Első és legfontosabb beállítás amit el kell végezni, hogy a hallgatóság az esemény során ülni vagy állni fog-e.
Lehetőségünk van a nézőteret különböző részekre is osztani, amennyiben a rendezvény során különböző helyeken
eltérő típusú részeket szeretnénk egyenletesen lefedni. Lehetőség van egyedi magasság beállítására is,
de jelen esetben a közönség egyhangúan állva fogja hallgatni a produkciót, ezért a \textit{``Standing''} opciót választottam.
Az előző lépésben elkészített rajzunkon definiálhatunk a program számára három fő régiót.
\newline Ezek az alábbiak:
%----------------------------------------------------------------------------
\begin{itemize}
	\item \textit{``Non Audience''} - a közönség területén kívül eső terület
	\item \textit{``Audience''} - a közönség területe
	\item \textit{``Hard Avoid''} - a közönség területén kívül eső terület, ahol a hangvisszaverődést szeretnénk minimalizálni
\end{itemize}
%----------------------------------------------------------------------------
Jelen esetben a fő hangrendszernél nem jelöltem meg a \textit{``Hard Avoid''} területet, mivel a teremben az első olyan felület
ami a hangvisszaverődést okozna már olyan távol helyezkedik el a hangrendszertől, hogy a hangvisszaverődés már nem okoz problémát.
A következő lépéseket előkészítve meg kell határoznunk a hangrendszertől egy adott távolságra lévő pontot a teremben,
amit referencia pontként fogunk használni. Ezt a pontot a \textit{``Move Ref''} gombra kattintva tudjuk megadni,
vagy manuálisan beírva az X és Y koordinátákat. Automatikusan a terem közepére van pozícionálva a referencia pont, de
ezt erősen ajánlott mozgatni attól függően mit szeretnénk elérni. Jelen esetben a mix pultot fogjuk a referencia pontnak megadni.
A \textit{``Start''} és \textit{``Stop''} mezőkben meg kell adnunk, hogy a referencia ponttól véve mekkora hangnyomás
deltával szeretnénk dolgozni. Ez azt jelenti, hogy a kezdő, a referencia és a végpont közötti hangnyomás hány dB-el térhet el egymástól.
Ezt az értéket a szoftver az eddig megadott információk alapján automatikusan kiszámolja, de manuálisan is megadhatjuk.
Az automatikus számítás az esetek többségében megfelelő eredményt ad, ezért most is ezt választottam.
A \textit{``Target SPL''} mezőben megadhatjuk a referencia ponton elérni kívánt hangnyomás szintet.
Így a rendszer \textit{``Gain''} struktúrája úgy lesz beállítva, hogy a referencia ponton 0 dBu bemeneti szint mellett elérjük a megadott hangnyomás szintet.
Magas frekvenciák csökkennek ahogy a távolság nő a forrástól, azaz a hangrendszertől.
Ha egyenletes frekvencia választást szeretnénk elérni nagyobb távolságokon, akkor a rendszernek nagyobb energiára lenne szüksége
a magas frekvenciákon, és kifutna a dinamika tartalékból, ezért jobb megoldás, ha a magas frekvenciák fokozatosan csökkennek a távolság növekedésével.
Beállíthatjuk a levegő veszteség kompenzációját, teljesen balra állítva nincs kompenzáció (figyelmen kívül hagyva a levegő abszorpcióját).
Teljesen jobbra állítva a maximális kompenzáció (a rendszernek 17dB headroom-ra van szüksége, hogy egyenes választ kapjunk).
Viszont ezekből az következik, hogyha túlságosan sok a kompenzáció, akkor a rendszernek nem lesz elég dinamika tartaléka, és a hang torzulni fog.
A változások hatását a \textit{``Target Response''} ábrán láthatjuk.
Ahhoz, hogy a számítások pontosak legyenek, elengedhetetlen, hogy pontosan megadjuk a környezeti változókat,
a hőmérsékletet, a páratartalmat és a légnyomást. Ezeket a paramétereket a \textit{``Edit''} gombra kattintva tudjuk megadni.
Jelen esetben mivel a teremben alapból is meleg van, a mérés időpontjában 28 fok, és a rendezvény során a közönség is melegíti a termet,
ezért a hőmérsékletet harminc fokra állítottam.
A páratartalom értéke a méréskor 57\%-os volt, de én 65\%-ra állítottam, mivel a rendezvény során a közönség által kibocsátott
vízgőz miatt a páratartalom nagy valószínűséggel magasabb lesz ennél. A légnyomás értékét pedig a helyi időjárás jelentésből vettem, ami
azon a napon 101800 Pa volt. Ezek beállítása után mivel az adott hangládához a gyári beállítások nagyon jók, ezért nem változtattam rajtuk,
a 14-es érték egyenletes és dinamikus hangvisszaadást biztosít.
%----------------------------------------------------------------------------
\begin{figure}[H]
	\centering
	\includegraphics[width=\textwidth, keepaspectratio]{figures/display_wpc_2.png}
	\caption{Display 2.3.4 b1 \textit{``Cover''} kezelőfelülete (WPC)}\label{fig:display_wpc_2}
\end{figure}
%----------------------------------------------------------------------------
Miután a \textit{``Cover''} kezelőfelületen elvégeztük a szükséges beállításokat, a \textit{``Splay''} kezelőfelületen folytatjuk a tervezést.
Az optimalizáció ezen részén a hangrendszert fogjuk a hallgatóság területére irányítani, a fokolási szögek beállításával.
A szoftver által biztosított optimalizációs algoritmus a lehető legjobb lefedettségre törekszik a tervezett területen.
Lehetőség van az optimalizáció súlyozási tényezőinek beállítására, de jelen esetben a gyári beállításokat használtam.
Amennyiben módosítani szeretnénk a súlyozást a \textit{``Target'} és a \textit{``Leakage''} mezőkben tudjuk megadni a súlyozási tényezőket.
A \textit{``Target''} mezőben megadott érték a közönség terület súlyozása,
a \textit{``Leakage''} mezőben megadott érték pedig a közönség területén kívül eső szivárgás súlyozása.
Az \textit{``Alow Polish''} opció engedélyezi a szoftvernek, hogy egy második körben finom hangolja a splay szögeket az első próbálkozás után.
Ezt az opciót előnyös bekapcsolni, mivel a szoftver így pontosabb eredményt tud produkálni, ezért ezt a beállítást mindig használom.
A \textit{``Max Time''} mezőben megadhatjuk, hogy a szoftvernek mennyi idő álljon rendelkezésére az optimalizáció elvégzéséhez.
Mivel a mai modern számítógépek olyan gyorsak, hogy a szoftver általában 1-2 perc alatt elvégzi az optimalizációt, ezért ezt az értéket
nem szoktam módosítani. A \textit{``Max Time''} mezőben megadott érték másodpercben értendő.
%----------------------------------------------------------------------------
\begin{figure}[H]
	\centering
	\includegraphics[width=\textwidth, keepaspectratio]{figures/display_wpc_3.png}
	\caption{Display 2.3.4 b1 \textit{``Splay''} kezelőfelülete (WPC)}\label{fig:display_wpc_3}
\end{figure}
%----------------------------------------------------------------------------
A következő lépés a \textit{``Rig''} kezelőfelületen történik. Ez a felület elsősorban az eddig elkészített
rendszerünket fogja megjeleníteni térben. Elsődleges beállítási paraméter ezen a panelen, hogy egy vagy két pontos
rögzítést szeretnénk-e használni. Jelen esetben egy pontos  rögzítést fogunk használni. Amennyiben valamilyen okból
szeretnénk változtatni a rendszer fizikai elhelyezkedésén, még megtehetjük, de ez a lépés ezen a ponton már nem ajánlott.
Bármely kis apró változtatás kardinálisan más végeredményhez vezethet. A tervezési folyamatot újra kell kezdeni, ellenkező esetben
a szoftver nem fogja tudni a megfelelő eredményt produkálni, és a rendszerünk nem úgy fog viselkedni a valóságban, ahogy azt mi szeretnénk.
A hangrendszer függesztéséhez és összeszereléséhez az összes információ megtalálható itt. Gondolva itt a riggvas fokolási helyére,
a ládák közti szögekre, a rendszer legfelső és legalsó pontjára.
Ezeken az információkon kívül még a rendszer súlyát és súlypontját is megkapjuk.
Esetünkben a teljes súly 289 kilogramm, amit a csarnok tartószerkezete biztonságosan elbír, valamint az egy tonnás emelőkapacitású
láncos emelők is képesek biztonságosan emelni. A súlypont a rendszer relatíve közepén helyezkedik el, ami stabil függesztést tesz lehetővé.
Ezek után a rendszert az említett paraméterek alapján össze építjük, figyelve az összes program által megadott információra.
%----------------------------------------------------------------------------
\begin{figure}[H]
	\centering
	\includegraphics[width=\textwidth, keepaspectratio]{figures/display_wpc_4.png}
	\caption{Display 2.3.4 b1 \textit{``Rig''} kezelőfelülete (WPC)}\label{fig:display_wpc_4}
\end{figure}
%----------------------------------------------------------------------------
Az utolsó lépés mielőtt ki tudnánk menteni a tervezett rendszert, az a \textit{``EQ''} kezelőfelületen történik.
Ha a \textit{``Cover''} kezelőfelületen már megadtuk a környezeti változókat, akkor ezt már nem kell újra megtennünk,
mivel a szoftver automatikusan átveszi az ott megadott értékeket. Beállíthatjuk az alsó és felső határfrekvenciákat, de mivel
a program a kiválasztott hangrendszerhez tartozó gyári beállításokat automatikusan betölti, ezért ezeket az értékeket sem kell módosítani.
Amit viszont érdemes és erősen ajánlott módosítani, az a \textit{``Freq Res''} és a \textit{``Space Res''} értékek. Az előbbi a
frekvencia felbontást, az utóbbi pedig a térbeli felbontást jelenti. Ezek az értékek határozzák meg, hogy a szoftver milyen 
felbontásban végezze el a számításokat. Minél kisebb értéket adunk meg, annál pontosabb eredményt fogunk kapni, viszont a számítások
hosszabb ideig fognak tartani. A \textit{``Freq Res''} értékét 1-re, a \textit{``Space Res''} értékét pedig szintén 1-re állítottam, mivel
ez az elérhető legnagyobb felbontás, és a számításokat is a lehető legpontosabban szeretném elvégezni. A gyári érték mindegyiknél a kettő.
Minél pontosabbak a számítások, annál jobban fog viselkedni a rendszer a valóságban és egyenletesebb hangvisszaadást fog produkálni.
Ha már a kiegyensúlyozott hangvisszaadásnál tartunk, akkor a \textit{``Resolution''} panelen meg kell adnunk, hogy milyen
konfigurációban szeretnénk használni a rendszert. A WPC szériás hangládákat tudjuk akár egyesével hajtani, azaz egy láda egy végfok csatorna
párral (mivel Bi-Amp hangládáról beszélünk). De a gyártó lehetőséget biztosít arra is, hogy a ládákat párosával vagy hármasával is hajtsuk.
Ennek költséghatékonysági és rugalmassági előnyei vannak, viszont a hangvisszaadás kevésbé lesz egyenletes. Jelen esetben 
az arany középutat választottam, és párosával fogom hajtani a ládákat. Így a WPC rendszer összesen 4 darab iK42 végfokot fog igényelni, ami 
16 darab végfokcsatornát jelent. A WPC rendszer csak iK42 végfokkal hajtható, a 8 csatornás iK81 végfokkal nem kompatibilis.
Lehetőség van az optimalizációs algoritmus befolyásolására is, a három előbbiekben már definiált súlyozási tényezők segítségével.
A fő hangsúlyt a \textit{``Target''} súlyozásra helyeztem, mivel a közönség területén szeretném a lehető legjobb hangvisszaadást elérni,
ezért 60\%-os súlyozást adtam neki.
A \textit{``Hard Avoid''} és a \textit{``Leakage''} súlyozását 20\%-ra állítottam.
Továbbá mindhárom résznek megadhatjuk mekkora hangnyomás értéket szeretnénk elérni a tervezett területen. Ezeket az értékeket
nem szükséges módosítani, mivel a gyári értékek megfelelőek, de ha mégis szeretnénk, akkor megtehetjük.
A paraméterezés után az optimalizáció után megkapjuk a teljes rendszer EQ beállítását és vizuális ábrázolást kapunk 
a referencia értéktől való eltérésekről.
%----------------------------------------------------------------------------
\begin{figure}[H]
	\centering
	\includegraphics[width=\textwidth, keepaspectratio]{figures/display_wpc_5.png}
	\caption{Display 2.3.4 b1 \textit{``EQ''} kezelőfelülete (WPC)}\label{fig:display_wpc_5}
\end{figure}
%----------------------------------------------------------------------------
%Amennyiben szeretnénk megtekinteni a tervezett rendszer teljesítményét amit a program számított ki, 
%akkor az \textit{``SPL''} kezelőfelületen, az \textit{``Index Plot''}-on a bal egeret nyomva tartva
%tudunk virtuálisan mozogni a teremben, és megtekinthetjük a számított hangnyomás és frekvencia eloszlás értékeket.
%Ha mindent rendben találunk és nincsenek kivetnivalóink a tervezett rendszerrel kapcsolatban, akkor
%sikeresen megterveztük a hangrendszert, és elmenthetjük a projektet.
%A mentés során egy MAT kiterjesztésű fájlt fogunk kapni, segítségével később bármikor újra megnyithatjuk a projektet,
%ha ugyan azon a helyszínen dolgozunk, gyorsabban és egyszerűbben tudjuk a rendszert újra tervezni, a szükséges
%módosításokat elvégezni.
%----------------------------------------------------------------------------
%\begin{figure}[H]
%	\centering
%	\includegraphics[width=\textwidth, keepaspectratio]{figures/display_wpc_6.png}
%	\caption{Display 2.3.4 b1 \textit{``SPL''}kezelőfelülete (WPC)}\label{fig:display_wpc_6}
%\end{figure}
%----------------------------------------------------------------------------
Az utolsó dolgunk ebben a programban mielőtt tovább lépünk, hogy exportáljuk a tervezett rendszert.
Az exportálás során egy D2P kiterjesztésű fájlt fogunk kapni, amit a VU-NET szoftver fog tud majd importálni a későbbiekben.
%----------------------------------------------------------------------------
\begin{figure}[H]
	\centering
	\includegraphics[width=\textwidth, keepaspectratio]{figures/display_wpc_7.png}
	\caption{Display 2.3.4 b1 exportáló kezelőfelülete (WPC)}\label{fig:display_wpc_7}
\end{figure}
%----------------------------------------------------------------------------
A fő hangrendszer megtervezése után a következő részegység aminek a tervét el kell készíteni, az a Delay hangrendszer.
A Delay hangrendszer a fő hangrendszerrel együtt fog működni, és a közönségtér hátsó-közép részétől kezdve fogja kiegészíteni azt.
Erre azért van szükség, mert a csarnokban a közönség ezen része olyan távolságra helyezkedik el, hogy
a WPC rendszer már nem tudja a megfelelő hangnyomás szintet egyenletesen biztosítani.
Ezt a feladatot a Wavefront Precision sorozatból a WPM típusú hangládák fogják ellátni, oldalanként 6-6 darab LineArray modullal.
Ez a láda egy két utas passzív hangrendszer, 2 darab 6.5"-os mély hangszóróval (LF) és 3 darab 1.4"-es magas hangszóróval (HF). 
Maximásan 130 dB hangnyomás szintet tud biztosítani, nagy előnye ennek a fajta rendszernek a súly-teljesítmény aránya, mivel egy
darab láda mindössze 14 kilogramm.~\cite{WPMUSERGUIDE}
%----------------------------------------------------------------------------
\begin{figure}[H]
	\centering
	\includegraphics[width=80mm, keepaspectratio]{figures/wpm_front_view.jpg}
	\caption{Martin Audio WPM LineArray modul}\label{fig:wpm}
\end{figure}
%----------------------------------------------------------------------------
A tervezési fázisok nagy része megegyezik az előbbi rendszer tervezésével, ezért ezeket a részeket nem ismétlem meg.
A hangsúlyt a eltérésekre helyezem, és azokat fogom részletezni.
A fő különbség a \textit{``Cover''} kezelőfelületen történik, ahol a \textit{``Hard Avoid''} területet kell megjelölni.
Ezen a rajzon már radikálisan szükség van erre a funkcióra, mivel a közönség területén kívül eső területen beton nagy felületek találhatóak,
amelyek jelentős hangvisszaverődést okoznának. 
%----------------------------------------------------------------------------
%\begin{figure}[H]
%	\centering
%	\includegraphics[width=\textwidth, keepaspectratio]{figures/display_wpm_1.jpg}
%	\caption{Display 2.3.4 b1 \textit{``Cover''} kezelőfelülete (WPM)}\label{fig:display_wpm_1}
%\end{figure}
%----------------------------------------------------------------------------
Ebből kifolyólag, szépen látható, hogy a program pontosan úgy optimalizálja a rendszert, hogy a \textit{``Hard Avoid''} területre, minél kevesebb
hangnyomás jusson.
Már a kék színnel jelölt terület első pár méterén radikálisan csökken a hangnyomás, és a rendszer a lehető legkevesebb energiát fordítja erre a területre.
A gyártó a WMP rendszert végfog csatornák szempontjából úgy tervezte, hogy a költség és a rugalmasság szempontjából akár négyesével is hajthatóak legyenek.
Ez persze nem jár kompromisszumok nélkül, a hangvisszaadás egyenletesebb lenne, ha egyesével hajtanánk a ládákat, de a jelenlegi rendszerben
hármasával fogom hajtani a ládákat, mivel ez a legköltséghatékonyabb megoldás, és végeredményben így is kielégítő hangvisszaadást fog produkálni mint 
kiegészítő egység. A WPM-eket tudjuk egyesével is hajtani, mivel rendelkezésünkre áll 2 db iK81 végfok.
%----------------------------------------------------------------------------
\begin{figure}[H]
	\centering
	\includegraphics[width=\textwidth, keepaspectratio]{figures/display_wpm_2.png}
	\caption{Display 2.3.4 b1 \textit{``Splay''} kezelőfelülete (WPM)}\label{fig:display_wpm_2}
\end{figure}
%----------------------------------------------------------------------------
Az előbbiekben már tárgyalt exportáló felületen az összes többi optimalizációs lépést követően
mentjük az elkészített tervet a WPC rendszerhez hasonlóan.
%----------------------------------------------------------------------------
\subsubsection{Martin Audio VU-NET rendszer szoftver~\cite{VUNETUSERGUIDE}}
%----------------------------------------------------------------------------
Első és legfontosabb feladatunk, hogy az összes végfoknak egyedi IP címet adjunk a hálózaton.
Enélkül a rendszerünk használhatatlan lesz, mivel a VU-NET szoftver nem fog tudni kommunikálni a végfokokkal.
Ezután győződjünk meg, hogy számítógépünk ugyan abban az alhálózatban és cím tartományban van mint a többi eszköz.
Miután már rendelkezünk az összes számunkra szükséges rendszer patch fájljával,
el tudjuk kezdeni felütni a végfokparkot amelyek a hangszórókat fogják hajtani.
Tehát a következő lépések a VU-NET szoftverben történnek a végfokok beállításával.
A VU-NET szoftver egy olyan alkalmazás, amely lehetővé teszi a Martin Audio hangrendszerek teljes körű vezérlését és monitorozását.
Jelen esetben az iK42 és iK81 típusú eszközöket fogjuk tudni kezelni.
Rendszerünk ha mindent összeszámolunk, akkor 8 darab iK42 és 2 darab iK81 végfokot fog tartalmazni.
Ebből 4 db iK42 a WPC LineArray rendszert, 2 db iK81 a WPM LineArrayt, 3 db iK42
az SX218 mélyládákat, és 1 db iK42 a FrontFill hangfalakat (XD12) fogja hajtani.
A Discover Devices gombra kattintva a szoftver megkeresi az összes végfokot a hálózaton, és megjeleníti azokat a
cím szerint növekvő sorrendben. Szinkronizáció után szabadon vezérelhetővé válnak az eszközök.
Egy kiválasztott eszközre jobb egérgombbal kattintva megjelenik egy menü, ahol az Open Preset Manager opcióra kattintva
betölthetjük az előre elkészített és gyári preseteket.
Az alábbi képen látható az a felület ahol bele kell tölteni a preseteket az egyes erősítőkbe külön-külön.
%----------------------------------------------------------------------------
\begin{figure}[H]
	\centering
	\includegraphics[width=\textwidth, keepaspectratio]{figures/vunet_systemdiagram_overall.png}
	\caption{VU-NET rendszer áttekintő diagram}\label{fig:vunet_systemdiagram_overall}
\end{figure}
%----------------------------------------------------------------------------
Így már szépen látható, hogy a rendszerünk hogyan fog kinézni, és melyik végfok melyik csatornán mit fog hajtani.
Ez kábelezés és rendszerstruktúra szempontjából is nagyon fontos információ, mivel kizárólag a megfelelő helyre
kötött hangládákkal fog megszólalni helyesen a rendszer. Ezen felül ha valamit rossz helyre kábelezünk le, tegyük fel egy
mélyládát hajtó végfokra egy Line Array modult, akkor a benne lévő hangszórók nagy valószínűséggel tönkremennek ha terhelés alá kerülnek.
Tehát fontos az precíz és átgondolt munkavégzés. A hibák elkerülése után is hasznos számunkra a rendszer áttekintő diagram,
mivel ha valami nem működik a rendszerben, akkor könnyen és gyorsan megtalálhatjuk a hibát, és azonosíthatjuk a hibás komponenst.
Miután sikeresen betöltöttük az összes presetet amire szükségünk van, be kell állítanunk, hogy az erősítők milyen forrásból
fognak jelet kapni. A következő lépésben a routing fülön be kell állítanunk, hogy az input csatorna Dante hálózatról fogja kapni a jelet.
Ez a beállítás a következőképpen néz ki:
%----------------------------------------------------------------------------
\begin{figure}[H]
	\centering
	\includegraphics[width=\textwidth, keepaspectratio]{figures/vunet_routing_dante.png}
	\caption{VU-NET Dante routing}\label{fig:vunet_routing_dante}
\end{figure}
%----------------------------------------------------------------------------
Most, hogy a megfelelő alapbeállításokat elvégeztük, a rendszer készen áll arra, hogy zajjal, például
Pink Noise-al teszteljük. Ezalatt a rendszer minden egyes komponensét külön-külön lehallgatjuk, és ellenőrizzük,
hogy minden egyes hangszóró megfelelően működik-e. Ha rendellenességet észlelünk, akkor azonnal kikapcsoljuk a rendszert,
és megnézzük, hogy mi okozza a problémát. Ha a probléma nem oldható meg, akkor a rendszert nem szabad tovább használni,
és az adott komponenst cserélni kell. Ha minden rendben van, akkor a rendszer készen áll a további feladatokra, mivel
még koránt sem értünk a végére a folyamatnak. A programba a mérési és optimalizálási folyamat közben még sok beállítást kell
elvégezni, amelyekről a későbbiekben lesz szó.
%----------------------------------------------------------------------------
\subsection{Dante hálózat kialakítása és optimalizálása}
%----------------------------------------------------------------------------
\subsubsection{Dante Controller}
%---------------------------------------------------------------------------- 
% Eszköz nézet
%----------------------------------------------------------------------------
Mielőtt neki állnánk konfigurálni az adott eszközt, fontos eldöntenünk, hogy
milyen módban szeretnénk használni.
Lehetőségünk van két fő mód közül választani, a redundáns és a
váltott mód közül. A \textit{``redundant''} mód mint ahogy azt a neve is sugallja
redundáns kommunikációt valósít meg az eszközök között szoftveresen és
hardveresen egyaránt. Az összes Dante kártya a jelenlegi rendszerben gyári konfigurációban két RJ45-s
csatlakozóval rendelkezik. Jelen esetben ezt a módot választjuk az
üzembiztosság és a kritikus hibák minimalizálása miatt.
A másik lehetőség a \textit{``switched''} pedig eszközök láncolását
teszi egyszerűbbé. Amennyiben a redundancia nem elsődleges szempont számunkra, nem kell
minden egyes eszköz mögé switch, hanem a másodlagos RJ45 port direktbe köti
az arra csatlakoztatott eszközt az elsődleges hálózatra. Így gyorsabban és
költséghatékonyabban tudjuk kiépíteni a hálózatot, azonban a redundancia lehetősége megszűnik.
%----------------------------------------------------------------------------
\begin{figure}[H]
	\centering
	\includegraphics[width=\textwidth, keepaspectratio]{figures/dante_devices.jpg}
	\caption{Dante hálózat állapot nézet}\label{fig:dante_devices}
\end{figure}
%----------------------------------------------------------------------------
\subsubsection{IP kiosztás}
%----------------------------------------------------------------------------
A rendszer képes automatikusan IP címeket osztani az egyes eszközöknek,
ezzel meggyorsítva a munkafolyamatot. Viszont ez nem bizonyul jó megoldásnak.
Egy fixen előre megtervezett rendszer praktikusabb és
üzembiztosabb megoldás, ha minden eszköznek manuálisan mi adjuk meg a címét a
hálózaton. A tervezett rendszerben minden egyes eszköznek fix IP címet adtam,
hogy könnyen és logikusan átlátható legyen az előbb említett előnyökön kívül.
A címeket egy online is elérhető Excel táblázatban tároltam, hogy amennyiben szükség van rá
bármikor könnyen elérhető legyen. Ez a táblázat a cégnél dolgozó összes munkatárs számára látható,
aki a rendszerrel foglalkozik. Így amennyiben új eszköz kerül a hálózatra, vagy egy eszköz IP címét
valamilyen okból meg kell változtatni, egyszerűen elérhető a szükséges naprakész információ.
A kiosztás logikája a következőképpen néz ki:
A Dante Primary hálózat a 192.168.1.X címeket használja, a Secondary hálózat
pedig a 192.168.2.X címeket. A két hálózat között nincsen semmilyen kapcsolat, és egymástól teljesen függetlenek hardveresen és szoftveresen is.
A Vu-Net vezérlés a 192.168.100.X címeket használja, és egy switchen keresztül üzemel a secondary hálózaton, viszont a két hálózat között nincsen
átfedés, az előbbiekben már említett módon függetlenek egymástól.
Az eszközök egyedi címei pedig az alábbi módon kerültek kiosztásra:
Vegyük példának a 192.168.1.111-es címet, a 111-ben az első számjegy arra utal, hogy egy végfokról van szó, minden végfok
1XX címet kap. A második számjegy az eszköz rack száma, a harmadik pedig az eszköz sorszáma a rackben felülről lefelé.
Tehát az előbbi cím a következőt jelenti számunkra: Az 1-es sorszámú végfokrackben lévő legfelső végfok.
A keverőpultok a címezés elején 1-től indulva 20-ig kapnak címeket. A stageboxok 30-tól 50-ig kapnak címeket.
Ezeken felül a háló legvégére vannak kiosztva a speciális eszközök melyek nem mindig vannak a rendszerben, de ha mégis
akkor azoknak is megvan a saját címük.
A háló 250-es címén helyezkedik el a Dante Audio szerver.
%----------------------------------------------------------------------------
\begin{figure}[H]
	\centering
	\includegraphics[width=\textwidth, keepaspectratio]{figures/dante_ips.png}
	\caption{Dante eszközök IP címei a hálózaton}\label{fig:dante_ips}
\end{figure}
%----------------------------------------------------------------------------
% Hálózati mátrix
%----------------------------------------------------------------------------
Ezen a felületen tudjuk a hálózaton összekapcsolni a különböző hang vevőket és
adókat. Egy nagy rendszerben a konfigurálása rendkívül nagy odafigyelést és
precíziót igényel, pontosan tudnunk kell mit, hogyan és miért kötünk össze.
Amennyiben hibásan konfiguráljuk a hálózatot, rendellenességek léphetnek fel a
hangrendszerben, amelyeket később nagyon nehéz és időigényes lehet kijavítani.
Hibás konfiguráció esetén előfordulhat, hogy egyes kimenetek máshol, vagy egyáltalán nem érkeznek meg a
végpontokhoz. Amint az alábbi képen látható, és a korábbiakban már említett 64x64-es kimeneti és bemeneti
mátrix-al tudunk garázdálkodni. Mivel egy általam tervezett produkcióban mindig szükség van 
egy L-R, egy SUB, és egy mono mátrixra, ezért a hálózatot a végéről kezdem el szaturálni.
A L a 61-es, a R a 62-es, a SUB a 63-as, a mono mátrix pedig a 64-es kimenetekre lesz kötve.
Így az állandó kimenetek átláthatóan és logikusan lesznek elrendezve, és nem kell
mindig keresgélni, hogy melyik kimenet melyik eszközhöz tartozik.
Mivel a jelenlegi rendszerben 2 darab DT 168-es stagebox található, ezért az 
1-16 kimeneteket a két stageboxra kötöm, logikusan 1-8 a 02-es sorszámúra és 8-16 a 03-as sorszámúra.
Ezek lesznek majd a fülmonitorok kimeneti pontjai, és a színpadon lévő zenészeknek fogják a jelet továbbítani.
(A fülmonitorokat ezen a rendezvényen a zenészek saját maguk biztosították, és  csak a jelet kellett továbbítani.)
Ezzel az FOH keverőnket sikeresen összekötöttük végfokokkal, és a monitor keverőnket a fülmonitorokkal.
Viszont még nem tudunk egyetlen egy bementet sem kezelni, mivel a keverők bemeneti mátrixa még üres.
%----------------------------------------------------------------------------
\begin{figure}[H]
	\centering
	\includegraphics[width=\textwidth, keepaspectratio]{figures/dante_pa_patch.jpg}
	\caption{Hálózati mátrix - Végfok patch}\label{fig:dante_pa_patch}
\end{figure}
%----------------------------------------------------------------------------
A rendszerbe integrált audioserver felelős a jelek elsődleges fogadásáráért és továbbításáért.
A DT168-as stageboxokba beérkező hangot először továbbítjuk az audioserverre, ahol aztán a
szükséges feldolgozásokat elvégezve továbbítja a jelet az FOH keverő felé.
A monitor keverőnek is szüksége van a jelekre, ezért a stageboxokból érkező jeleket szintén továbbítjuk
de ebben az esetben direktben a monitor keverőbe, az audioserver közbeiktatása nélkül.
Ez azért fontos mert az audioserver egy minimálisan késleltetett jelet továbbít, ami a monitor keverőnél
nem kívánatos, mivel a zenészeknek a lehető legkevesebb késleltetésre van szükségük. Erről a későbbiekben a méréseknél
bővebben lesz szó. 
Így most már a 32 bemenetünket is tudjuk használni, és a rendszerünk készen áll a további konfigurációkra.
%----------------------------------------------------------------------------
\begin{figure}[H]
	\centering
	\includegraphics[width=\textwidth, keepaspectratio]{figures/dante_audioserver_patch.jpg}
	\caption{Hálózati mátrix - Audioserver patch}\label{fig:dante_audioserver_patch}
\end{figure}
%----------------------------------------------------------------------------
% Órajel nézet és késleltetés
%----------------------------------------------------------------------------
A következőkben meg kell adnunk az audio hálózatunk master órajelét. Ehhez az órajelhez
szinkronizál a többi eszközünk a hálózaton.
Az időszinkronizáció kulcsfontosságú élőzenei produkcióknál,
a mi esetünkben a keverőpult lesz a master órajel, ő fogja a hálózatot
vezérelni. A felületen egyszerűen bepipáljuk a \textit{``Preferred leader''} opciót
a keverőpult mellett, és a hálózat többi eszköze automatikusan ehhez az órajelhez szinkronizálódik.
Az órajelen kívül a késleltetési értékeket is be kell állítanunk. A default érték minden eszköznél jelen esetben
1 ms, mivel ennél az értéknél a legtöbb eszköz képes probléma nélkül működni. Ez az érték azonban
nem minden esetben optimális, ezért fontos, hogy minden eszköz késleltetését ellenőrizzük és beállítsuk.
Amennyiben problémákat, zavaró pattanásokat vagy hangkimaradásokat tapasztalunk, akkor a késleltetési értékeket
növelni kell mindaddig, amíg a probléma nem szűnik meg. Fontos megjegyezni, hogy a késleltetési értékek
növelésével egyre több és több időt vesz igénybe a jelek feldolgozása, előzenei produkcióknál ez kritikus.
A mi esetünkben mivel az útválasztók is kifejezetten csak erre a célra vannak használva és semmiféle más
adatforgalmat nem bonyolítanak, a 1 ms-os késleltetési érték csökkenthető is akár a felére is.
Rövid kábelhosszak esetén egyes eszközök között akár 0.25 ms-os késleltetési értéket is beállíthatunk amennyiben
hosszútávon nagy biztonsággal stabil a hálózat. Jelen esetben a stabil működés érdekében 1 ms-os késleltetési értéket
tartottam meg.
%----------------------------------------------------------------------------
\subsubsection{Dante rendszer monitorozása}
%----------------------------------------------------------------------------
Miután végeztünk a hálózat konfigurációjával, a munkánk nem ér véget, hiszen
a rendszer működését folyamatosan monitorozni kell. 
Minden egyes eszköznél a lantecy oldalon láthatjuk az aktuális és átlagos késleltetési értékeket.
Amennyiben ez az érték a konfigurált késésnél alacsonyabb, akkor a rendszerünk
stabilan működik. Ha azonban magasabb, vagy néha drasztikusan közel kerül ehhez az értékhez akkor azonnal cselekednünk kell, mivel
azt kockáztatjuk, hogy a rendszerünk instabil lesz, és a produkció közben problémák léphetnek fel.
Egy élő produkciónál a hangkimaradás vagy kiszámíthatatlan pattanások megengedhetetlenek, és az egész rendezvényre 
rá nyomja a bélyegét. Miután az egyes eszközök értékeit ellenőriztük, átválthatunk egy átfogó hálózat nézetre,
a Network Status fülre, ahol az összes eszköz késleltetési értékeit egyszerre láthatjuk. Ezen kívül még elérhető
metrikák:
%----------------------------------------------------------------------------
\begin{itemize}
    \item Az elsődleges és másodlagos hálózaton haladó adatfolyam mennyisége és állapota
    \item Voltak-e csomagvesztések
    \item Egyes eszközök kapcsolásának állapota
\end{itemize}
%----------------------------------------------------------------------------
\begin{figure}[H]
	\centering
	\includegraphics[width=\textwidth, keepaspectratio]{figures/dante_latency.jpg}
	\caption{DT 168 stagebox késleltetési értékeinek monitorozása}\label{fig:dante_latency}
\end{figure}
%----------------------------------------------------------------------------
\newpage
%----------------------------------------------------------------------------
\subsection{Mélyláda rendszer}
%----------------------------------------------------------------------------
A mélyláda hangja hosszabb hullámhosszú, mint a többi komponensé, ezért az
optimális helymeghatározásuk és elhelyezésük kulcsfontosságú a megfelelő
hangzás érdekében. 
A rendszerben a Martin Audio SX218 típusú mélyládáit fogjuk használni.
Ezek a ládák dupla 18"-os mély hangszórókkal vannak felszerelve, 2000W AES és 8000W csúcsteljesítményre képesek, és
maximálisan 144 dB hangnyomás szintet tudnak biztosítani.~\cite{SXSUBWOOFERUSERGUIDE}
Jelen esetben 12 darab ilyen láda fogja biztosítani a megfelelő mély tartományt a rendezvényen.
%----------------------------------------------------------------------------
\begin{figure}[H]
    \centering
    \begin{minipage}{0.45\textwidth}
		\centering
		\includegraphics[width=150px, keepaspectratio]{figures/sx218_front_view.jpg}
		\caption{Martin Audio SX218}\label{fig:sx218}
    \end{minipage}\hfill
    \begin{minipage}{0.45\textwidth}
        \centering
        \includegraphics[width=\linewidth, keepaspectratio]{figures/sub_array_63hz.jpg}
        \caption{SX218 mélyláda rendszer}\label{fig:sub_array_63hz}
    \end{minipage}
\end{figure}
%----------------------------------------------------------------------------
A \textit{``SUB''} tervezést egy általam készített Excel kalkulátor
segítségével végzem el. Ez a táblázat egyesíti a Martin Audio és Merlin van Veen által
készített kalkulátorokat (S.A.D), valamint kiegészítésre került további modulokkal és funkciókkal.~\cite{MERLINVANVEEN}~\cite{MARTINSUBCALCULATOR}
A egy EndFire konfigurációs mélyláda elrendezést terveztem. 
(Egy EndFire Pack = két láda egymás előtt adott távolságra és késleltetési értékkel) 
A táblázatba megadhatjuk, hogy éppen helyileg hol
van a rendezvény, és egy időjárás API segítségével megkapjuk az adott napra/napokra a
teljes időjárás előrejelzést. Majd ezekből egy adott időintervallumra átlagolva
megkapjuk az éppen aktuális hőmérsékletet, páratartalmat, légnyomást, amiből kiszámolható
a hangsebesség, ezáltal még tovább optimalizálható a rendszer, mivel nem egy előre átlagot fix értéket
használunk, hanem az aktuális körülményekhez igazítjuk a rendszert.
Amennyiben a helyszínen rendelkezünk mérőeszközökkel, melyek képesek pontosan mérni a hőmérsékletet,
páratartalmat és légnyomást, akkor ezeket az értékeket is be tudjuk állítani a táblázatban.
A program kiszámolja, hogy milyen távolságra kell a ládákat helyezni egymástól előrefelé,
valamint mekkora \textit{``delay''} értéket kell alkalmazni. 
Majd az egyes EndFire Pack-ok egymáshoz képesti távolságot oldalirányban és azok
közötti delay értéket is megkapjuk. A \textit{``SUB array''}-t 63 Hz-re optimalizáltam, mivel
ez az a frekvenciatartomány, ahol a mélyláda rendszer a legtöbb energiát tudja leadni.
A program pontosan kiszámolja számunkra a beállított paramértereknek megfelelően hogyan kell megépíteni, 
vagyis fizikailag elhelyezni a ládákat, valamint az összes időzítési értéket is megkapjuk mellé.
A program hosszútávú használata során kiderült, hogy szinte század milliszekundum pontossággal
megkapjuk a várt eredményt amit méréssel is ellenőriztünk, és a két eredmény között aligha volt eltérés.
Ezzel a munkafolyamatunkat nagyban felgyorsítottuk, és a munkánk precizitását is megtartottuk.
%----------------------------------------------------------------------------
\begin{figure}[H]
	\centering
	\includegraphics[width=350px, keepaspectratio]{figures/endfire_excel.jpg}
	\caption{EndFire hangleképezés predikció}\label{fig:endfire_excel}
\end{figure}
%----------------------------------------------------------------------------
\subsection{Allen \& Heath digitális keverőrendszer}
%----------------------------------------------------------------------------
A jelenlegi rendszer két keverőpultot fog tartalmazni, egyet a fő hangrendszerhez, és egyet a monitor rendszerhez.
Mindkét keverőpult Allen \& Heath SQ-6 típusú digitális keverőpult lesz.
A pultok 96 kHz-es mintavételezési frekvenciával operálnak és 48 csatornát képesek maximálisan kezelni, melyek közül 
24 csatornával rendelkezik fizikailag beépített mikrofon előerősítővel. A konzolokon található 16 programozható gomb,
25 fader melyek 6 rétegben helyezkednek el és a felhasználó személyre szabhatja őket a saját igényei szerint. 
Emellett 12 sztereó mix áll a rendelkezésünkre, melyeket szintén a felhasználó konfigurálhat saját igényei szerint.
A sztereó mixek testreszabásával tudunk csoportokat is létrehozni.
A konzolokon 8 sztereó effekt motort is megtalálunk, ezekbe a virtuális effekt processzorokba a pulton található ingyenes és fizetős 
pluginokat tudjuk betölteni. (amennyiben megvásároltuk a fizetős csomagokat, jelen rendszerben ezek nincsenek megvásárolva)
További előnye a platformnak, hogy egy 32x32 csatornás USB audio interfésszel rendelkezik, így a számítógéphez csatlakoztatva
egy nagy felbontású hangkártyaként is használhatjuk.~\cite{AHSQ}
%----------------------------------------------------------------------------
\begin{figure}[H]
	\centering
	\includegraphics[width=70mm, keepaspectratio]{figures/sq6.jpg}
	\caption{Allen \& Heath SQ-6 digitális keverőpult}\label{fig:sq6}
\end{figure}
%----------------------------------------------------------------------------
Az I/O bővítőkártyák közül a rendszerben mindkét pultban megtalálható egy darab SQ Dante kártya ami 64x64 csatorna
kezelésére képes. Az A\&H SQ szériás pultjai csak egy darab I/O bővítőkártyát tudnak kezelni, de léteznek
olyan rendszerek mint például az Avantis és a dLive szériás pultok, amelyek több I/O bővítőkártyát is tudnak kezelni.
Jelen esetben ez teljes mértékben felesleges, mivel a rendszerben kizárólag a Dante protokollra támaszkodunk.
%----------------------------------------------------------------------------
\begin{figure}[H]
    \centering
    \begin{minipage}{0.45\textwidth}
        \centering
        \includegraphics[width=50mm, keepaspectratio]{figures/sq_dante.jpg}
        \caption{A\&H SQ Dante kártya}\label{fig:sq_dante}
    \end{minipage}\hfill
    \begin{minipage}{0.45\textwidth}
        \centering
        \includegraphics[width=50mm, keepaspectratio]{figures/dt168.jpg}
        \caption{A\&H DT168 Dante stagebox}\label{fig:dt168}
    \end{minipage}
\end{figure}
%----------------------------------------------------------------------------
\subsection{Dante audio szerver}
%----------------------------------------------------------------------------
Ez a kiegészítő szerver egység lehetővé teszi bármilyen alacsony késleltetéssel dolgozó VST3 plugin használatát a rendszerben elő környezetben.
Olyan komplex funkcionalitásokat is elérhetünk, amelyek a keverőpulton csak limitáltan, vagy egyáltalán nem elérhetőek.
Gondolva itt a dinamikus EQ használatára, különböző kompressziós technikákra (például Opto, Multiband Compressor), 
A legnagyobb előnye ennek a fajta megoldásnak, hogy költségek szempontjából egy magasabb kategóriás keverőpult rendszer sokkal drágább lenne,
valamint nem vagyunk korlátozva a keverőpult által biztosított funkcionalitásokkal, bármikor tudunk igényeink szerint újabb és újabb pluginokat
telepíteni a rendszerbe, amíg a számítógép hardveres erőforrásai ezt lehetővé teszik.
Az általam épített szerver egy AMD Ryzen 7950X processzorral és 32 GB DDR5 memóriával, és a Focusrite RedNet PCIe kártya veszi fel a harcot a
komplex hangfeldolgozási feladatokkal. Ezekre a hardverekre azért esett a választás, mert mivel a Dante kártya 128 csatornát tud kezelni, 
(az épített rendszer 64 csatornás, de a bővítés lehetősége fent áll) erős számítási kapacitásra van szükség, hogy a rendszer a beállított rendkívül alacsony
késleltetési értékek mellett is képes legyen késés nélkül a csatornák feldolgozására. Amennyiben nem sikerül a jelet a beállított időn belül
produkálni, furcsa zavaró pattogó hangokat hallhatunk, vagy rosszabb esetben hangkimaradás is előfordulhat. Ezért fontos a megfelelő hardveres
erőforrások biztosítása, és a pontos beállítások elvégzése.
A példában szereplő rendszer kifogástalanul képes elvégezni a feladatát, és a beállított 1 ms-os késleltetési értéket is képes folyamatosan tartani.
%----------------------------------------------------------------------------
\begin{figure}[H]
	\centering
	\includegraphics[width=55mm, keepaspectratio]{figures/rednet_pcie.jpg}
	\caption{Focusrite RedNet PCIe kártya}\label{fig:rednet_pcie}
\end{figure}
%----------------------------------------------------------------------------
Szoftveres oldalról a gépen egy speciális Windows 11 rendszer került telepítésre. A neve \textit{Ghost Spectre Windows 11 Superlite SE}, amely egy
teljesítmény és tárhely optimalizált Windows 11 verzió. A szakdolgozat írásakor elérhető legfrissebb stabil verzióját használtam fel, ami a
23H2-es 22631.3593 verziószámú.
A rendszerben csak a legszükségesebb alap szoftverek vannak telepítve, és a többi 
alkalmazás, amely nem szükséges a rendszer működéséhez, törölve lett. A lehető legkevesebb erőforrást használja, miközben egy stabil és megbízható
munkakörnyezetet biztosít. A frissítések automatikusan letiltásra kerültek, hogy a rendszer ne legyen kitéve a Windows frissítések által okozott
esetleges hibáknak, valamint a világhálóra való csatlakozás is letiltásra került. Kizárólag külső adathordozókról lehet adatokat átvinni a rendszerbe.
A számunkra szükséges drivereket és programokat kizárólag így tudjuk telepíteni a rendszerbe. 
Mivel az FOH pult Dante patchet nézve nem csatlakozik direktben a stageboxokhoz, hanem az audioserveren keresztül, ezért a stageboxok
előerősítőit az audioserveren keresztül tudjuk szabályozni. Erre a célra az Allen \& Heath DT Preamp Control nevű szoftverét használjuk,
amely a stageboxokat tudja szabályozni, valamint a stageboxok firmware frissítését is lehetővé teszi.
Minden egyes csatornára beállíthatjuk a kívánt Gain értéket, tudunk PAD-et kapcsolni, valamint a Phantom tápot is tudjuk be és kikapcsolni.
(A PAD egy olyan kapcsoló, amely a 20 dB-es csillapítást kapcsol a bemeneten, ha a bemeneti jel túl erős lenne, és ezáltal
elkerülhető a torzítás. A Phantom táp pedig a tápellátást igénylő mikrofonok számára szükséges 48V-ot biztosítja.)
%----------------------------------------------------------------------------
\begin{figure}[H]
	\centering
	\includegraphics[width=350px, keepaspectratio]{figures/dt_preamp_control.jpg}
	\caption{DT 168 stagebox előerősítő szabályozó felülete}\label{fig:dt_preamp_control}
\end{figure}
%----------------------------------------------------------------------------
A DAW szoftverek közül a rendszerben az audioström által fejlesztett Live Professor szoftver található, amely egy
VST host szoftver, amely képes több VST plugin egyidejű futtatására, és a pluginokat a felhasználó igényei szerint
rendszerezni, csoportosítani. Kifejezetten live használatra lett tervezve a kezdetektől fogva, ezáltal sok olyan
funkciót tartalmaz, amelyek egy FOH mérnök számára könnyebbé teszik a használatot. Például a pluginokat láncokba tudjuk
rendezni, és ezeket a láncokat egy gombnyomással be és kikapcsolni, automatizálni. MIDI vezérlésre is képes, így
például a keverőpulttal egy USB kábellel való összekötés után tudjuk vezérelni a szoftvertben lévő paramétereket a keverőpult soft gombjaival. (ezek személyre szabható gombok)
Mielőtt azonban használnánk a programot, fontos a megfelelő beállítások elvégzése, hogy a rendszerünk stabilan működjön.
Első és legfontosabb feladat az audio hardver és driver kiválasztása, valamint a megfelelő buffer méret beállítása a mintavételezési frekvenciával együtt.
Jelen esetben ez a RedNet PCIe kártya lesz, és a buffer méret a lehető legkisebb, 32 minta lesz, ASIO driverrel, 96 kHz-es mintavételezési frekvenciával.
Így mind a 128x128 csatorna máris a rendelkezésünkre áll a használatra.
A buffer méret azért lehet ennyire alacsony, mivel a számítógép hardveres erőforrásai ezt lehetővé teszik, és a rendszer
képes a 32 minta buffer méretet stabilan tartani. A buffer méret csökkentésével a késleltetés is csökken, 
azonban a processzor terhelése is nő, ezért fontos a megfelelő hardveres erőforrások biztosítása. 
A szoftverben beállíthatjuk hány párhuzamos szálon történjenek a számítások, mivel a rendszerben lévő processzor 24 szálas,
ezért a programot 24-re konfiguráljuk, így a processzor kihasználtsága optimális lesz. Valamint a Windows Scheduler Prioritását is
a legmagasabbra állítjuk, ami a Realtime Prioritás, így a rendszer a lehető legnagyobb prioritással fogja kezelni a programot.
Az alábbi képen látható egy minta plugin lánc, melyet egy vokál csatornát képvisel. A láncban található egy
SSLG Channel Strip, egy Primary Source Expander, egy CLA-76 Compressor, egy Waves Tune Real-Time és egy FabFilter Pro-Q3.
Ezek a pluginok együttesen képesek egy vokál csatornát teljesen átalakítani, és a kívánt hangzást elérni.
A felső sávban látható a Processing Time, ami azt mutatja, hogy a rendszer mennyi idő alatt képes a jelet feldolgozni.
Mellette látható a már említett 32 minta buffer méret, és a 96 kHz-es mintavételezési frekvencia.
Ami viszont számunkra a legfontosabb, a mellette lévő két kis téglalap. A Live Professor képes észlelni a hangkimaradásokat,
és ezeket a téglalapokat pirosra váltani, ha a rendszer nem képes a beállított buffer méretet stabilan tartani.
(Elsőnek narancssárga figyelmeztetést kapunk, ha a közel járunk a buffer méret határához.)
Mivel jelen esetben a rendszer stabilan működik, a téglalapok zöld színűek, és a Processing Time is a beállított értéken belül van.
%----------------------------------------------------------------------------
\begin{figure}[H]
	\centering
	\includegraphics[width=\textwidth, keepaspectratio]{figures/waves_plugins.jpg}
	\caption{Waves és FabFilter Audio pluginok}\label{fig:waves_plugins}
\end{figure}
%----------------------------------------------------------------------------
\subsection{A rendszer mérése}
%----------------------------------------------------------------------------
A rendszermérések elkészítésére a Rational Acoustics által fejlesztett Smaart nevű szoftvert fogom használni, mivel
az iparágban ez a legelterjedtebb és legmegbízhatóbb szoftver a mérések pontos elvégzésére.
A szoftver legújabb verzióját fogom használni, ami az írás pillanatában a 'Smaart Suite 9.4.1' verzió.
Első lépésben a számítógéphez csatlakoztatott hangkártyát kell konfigurálni, hogy el tudjunk kezdeni méréseket végezni.
Jelen esetben a hangkártya maga a keverőpult, aminek az USB interfésze egy 32x32-es kapcsolatra képes a számítógéppel.
A mérőmikrofon egy Behringer ECM8000 típusú mérőmikrofon lesz, ami XLR csatlakozóval kapcsolódik a keverőhöz, a mikrofon tápellátását
a keverőpult biztosítja 48V fantomtáppal. Annak érdekében, hogy a mérés működőképes legyen, a mikrofont a keverőpulton be kell szintezni, minden 
processzálást kikapcsolni. A mikrofon kimenetét utána Direct Out móddal el kell küldeni az USB interfészre, ahol a Smaart szoftver
fogja tudni a jelet feldolgozni. Ezen kívül egy másik bemenetre is szükség lesz a keverőpulton, ahol a Smaart szoftver a referencia jelet fogja
küldeni, ezt szintén Direct Out módban vissza kell küldeni a számítógépre.
Tehát a példa kedvéért még egyszer:
Local Input 1: Behringer ECM8000 mérőmikrofon, majd Direct Out Output USB 1-re.
USB Input 1: Smaart szoftver referencia jele, majd Direct Out Output USB 2-re.
Az első mérés amit végezni fogok a rendszeren a mélyláda rendszer mérése lesz.
Az EndFire konfigurációban elhelyezett mélyláda rendszer mérése során a cél az, hogy maximalizáljuk a hangnyomást a közönség távolabbi részein is,
miközben hátrafelé kioltást érünk el. Ezen felül fontos a rendszer nyitása is, ha túl keskeny sávban működik, akkor a középső területek
hangnyomása túl magas lesz, a szélső területeken pedig túl alacsony. Tehát fel van adva a lecke, hogy a rendszer a lehetőségekhez mérten
egyenletes hangnyomást biztosítson a teljes területen.
Fontos, hogy a mérési környezet a lehetőségekhez mérten minél csendesebb legyen, valamint
ne legyenek olyan tárgyak a nézőtéren, amelyek a rendszer elő működése közben
nem lesznek jelen és mérés közben a mért hangot visszaverik. Ezzel minimalizálhatjuk a
hibákat és a mérések pontatlanságát, valamint a mérés reprezentatívabb lesz.
%----------------------------------------------------------------------------
% SUB - TOP Align
%----------------------------------------------------------------------------
Mivel a mélyláda rendszer és a Main PA fizikai elhelyezkedésük miatt (A Main PA riggelve van, míg a mélyládák a földön helyezkednek el)
egy adott távolságra vannak egymástól, fontos, hogy a két rendszer fázishelyes és időhelyes legyen.
Amennyiben a rendszer fázishelytelen, a rendszer elemei egymás ellen dolgoznak, kioltást okozva, és ezzel csökkentve a hangnyomást.
Ezenkívül hiába van fázisban a rendszer, fontos, hogy a két rendszer időben is helyes legyen. Előfordulhat, hogy fázisban vagyunk de 180 fokkal el vagyunk tolva
az egyik irányba, így a mélyek vagy késnek vagy előbb érnek a hallgatóhoz ezzel rongálva a hangképet.
Aggodalomra azonban nincs ok, mivel megfelelő szakértelemmel és a megfelelő mérési eszközökkel ezek a problémák orvosolhatóak, ezzel biztosítva
a kiváló hangminőséget a rendezvényen.
Az alábbiakban a mérési eredmények láthatóak.
Fontos megjegyezni, hogy a mérési eredmények csak a mérés helyszínén érvényesek, és a mérési körülményektől függően változhatnak.
%A reflexiók és a mérési bizonytalanság mennyisége infidecimális a mérési eredményben.
Az ábrán a következőket láthatjuk különböző színekkel jelölve:
- A kék szín a teljes rendszer viselkedése mérés nélkül.
- A zöld szín a teljes rendszer viselkedése mérés után.
- A magenta szín csak a main PA rendszer viselkedése.
%----------------------------------------------------------------------------
\begin{figure}[H]
	\centering
	\includegraphics[width=\textwidth, keepaspectratio]{figures/smaart_sub_top.jpg}
	\caption{SUB - TOP mérések}\label{fig:smaart_sub_top}
\end{figure}
%----------------------------------------------------------------------------
Az ábrán egyértelműen látható, hogy a rendszer mérése után a hangnyomás szintje jelentősen nőtt, a crossover frekvencia környékén
57-87 Hz között. A kék mérésnél egy óriási beesést láthatunk 70 Hz-nél, ami azt jelzi számunkra, hogy a rendszer nincs fázisban és időben.
A korrekt időértékek beállítása után a zöld színű mérésnél egyértelműen látható, hogy a hangnyomás szintje jelentősen nőtt, és a 70 Hz-es
beesés teljes mértékben eltűnt, sőt a hangnyomás szintje a crossover frekvencia környékén jelentősen nőtt. Ez azt jelzi számunkra, hogy a rendszer
fázishelyes és időben is helyes már, készen áll a produkcióra.
%----------------------------------------------------------------------------
% FrontFill - TOP Align
%----------------------------------------------------------------------------
A frontfill mérését a Smaart-ba beépített delay finder segítségével fogom elvégezni.
Elsőnek elhelyezzük a mérőmikrofont a megfelelő helyen, ahová a main PA és a frontfill rendszer hangja is érkezik.
Ezután a delay finder segítségével elsőnek megmérjük a main PA rendszer hangját, majd ezek után
a frontfill rendszer hangját. A delay finder segítségével a két rendszer közötti késleltetési értéket
automatikusan kiszámolja a szoftver, és megkapjuk mint idő delta értéket.
Az alábbi képen látható, hogy a késleltetés delat értéke 15.41 ms, tehát a frontfill rendszert kell
15.41 ms-el késleltetni, hogy a két rendszer fázisban és időben is helyes legyen.
%----------------------------------------------------------------------------
\begin{figure}[H]
	\centering
	\includegraphics[width=300px, keepaspectratio]{figures/nearfill_smaart.jpg}
	\caption{FrontFill - TOP mérések}\label{fig:nearfill_smaart}
\end{figure}
%----------------------------------------------------------------------------
\subsection{A rendszer monitorozása}
%----------------------------------------------------------------------------
Az időzítési értékek megfelelő beállítása után a rendszer hangnyomásának monitorozása a fontos.
A rendszer monitorozására a Smaart szoftverben található SPL monitor funkciót fogom használni.
Ehhez szükséges egy kalibrált mikrofon, amelynek a kalibrációs értékeit a Smaart szoftverben be kell állítani. 
Ez általában egy 94 dB-es referencia hangnyomás szintet jelent 1 kHz-en, melyet egy mikrofon kalibrációs eszközzel tudunk elvégezni ami képes pontosan kiadni ezt a hangnyomás szintet.
Így a mérési értékek pontosak lesznek, és a rendszer hangnyomás szintjét valósan tudjuk mérni.
Többféle módszerrel lehet a hangnyomás szintet monitorozni, jelen esetben hármat választottam ki:
- dB SPL C Slow => A frekvencia súlyozott hangnyomás szintet mutatja, a mikrofon érzékenységéhez igazítva.
- dB SPL A Slow => A frekvencia súlyozott hangnyomás szintet mutatja, az emberi hallás érzékenységéhez igazítva.
- dB Leq 1 => A hangnyomás szint átlagértékét mutatja 1 perces időtartamra.
%----------------------------------------------------------------------------
\begin{figure}[H]
	\centering
	\includegraphics[width=100mm, keepaspectratio]{figures/smaart_spl_meter.jpg}
	\caption{SPL monitorozás a produkciók közben}\label{fig:smaart_spl_meter}
\end{figure}
%----------------------------------------------------------------------------
\chapter{\FurtherDevelopment}
%----------------------------------------------------------------------------
\section{Üzemeltetési tapasztalatok és a végeredmény} % Konkrétan mely helyszíneken, hogyan zajlott, több tapasztalat megosztása, mi működött jól, és mi kevésbé.
%----------------------------------------------------------------------------
Ahogy már a modellezés során a korábbiakban említettem, a budapesti Millenáris B csarnoka 
volt a referencia helyszín, ahol a rendszert élesben telepítettem egy rendezvény idejére.
A képek magukért beszélnek, a rendszer tökéletesen működött, a hangminőség kiváló volt, a hangnyomás szint pedig
a tervezett értékeknek megfelelően alakult. A mélyláda rendszer és a Main PA rendszer fázishelyes és időben is helyes
volt, a frontfill rendszer késleltetése is megfelelően beállításra került, a rendszer hangja kiegyensúlyozott volt.
A rendszer működése során nem voltak hibák, a rendszer stabilan működött
a teljes rendezvény alatt. 
Nem voltak csomagvesztések a hálózaton, egyetlen egyszer sem kellett a másodlagos redundáns hálózatra váltani probléma miatt.
A szoftveres megoldások is kiválóan működtek, a számítógép stabilan futott a
rendelkezésre álló hardveres erőforrásokkal. A mérések is pontosak voltak, a rendszer minden egységét
sikerült beállítani a megfelelő értékekre. 
A terem akusztikája alapvetően visszhangosnak mondható, mivel a teremben sok kemény felület található,
beton, fém és üveg felületek, amelyek a hangot visszaverik. Ehhez még hozzá járul a rendkívül magas belmagaasság is,
ami szintén nem könnyítette meg a munkát. A rendszer azonban jól teljesített, a visszhangot sikerült minimalizálni,
és a legutolsó sorban is tisztán és egyenletesen hallható volt.
A fellépő zenekar és lemezlovas is meg volt elégedve a rendszerrel, dicsérték a rendszert minden aspektusból.
Végül a megrendelő teljes mértékben elégedett volt a szolgáltatással, ezáltal 
a projekt sikeresnek mondható.
Összefoglalva minden komponens megfelelően végezte a rá bízott feladatot, és a rendszer
stabilan kihagyás nélkül működött a teljes üzemidő alatt. 
%----------------------------------------------------------------------------
\begin {figure}[H]
    \centering
    \includegraphics[width=300px]{figures/danci_wpc.jpg}
    \caption{A szakdolgozat szerzője a megépített WPC Line Array mögött emelés előtt}
\end {figure}
%----------------------------------------------------------------------------
\section{Továbbfejlesztési lehetőségek}
%----------------------------------------------------------------------------
\subsection{További eszközök integrálása}
%----------------------------------------------------------------------------
A Dante networking keresztrendszer lehetővé teszi a rendszer folyamatos bővítését a
hálózati limitációk megfelelő kezelésével. A rendszer bővítésekor figyelembe kell venni a
még rendelkezésre álló, a sávszélességet és a késleltetést mértékét. Amennyiben
tarjuk magunkat ezekhez a paraméterekhez, a rendszer bővítése nem okozhat problémát,
és megfelelő overhead mellett elméletileg a teljes hálózatot is szaturálhatjuk mindenféle probléma nélkül.
A Martin Audio Wavefront sorozatú hangfalak skálázható felbontása lehetővé teszi, hogy
a rendszer bővítésekor a már meglévő hangrendszerünket több végfokkal hajtva tovább
növeljük a rendszer teljesítőképességét. A korábbiakban már említett felbontás növelés
javítja a rendszer hangminőségét, frekvenciafelbontását és az adott területen való
pontosabb hangeloszlást. Ebből kifolyólag nagy fejlesztés lehet a jövőben a rendszer egy ládás
felbontásra való kibővítése. Ez azt jelenti, hogy az összes Line Array egységet külön-külön
végfok csatornával hajtjuk meg.
További fejlesztés a rendszerben a LineArray hangfalak számának növelése megfelelő számú végfok egységgel,
amely tovább növeli a rendszer teljesítményét és a maximálisan lefedhető területet.
%----------------------------------------------------------------------------
\subsubsection{Shure Axient Digital rendszer} % A jel egyből a Dante hálózaton keresztül érkezik a keverőbe nincs A/D konverzió
%----------------------------------------------------------------------------
A Shure Axient Digital vezeték nélküli mikrofon rendszer a szakma egyik legjobb vezeték nélküli
mikrofon rendszere. A rendszer kiváló hangminőséget, megbízhatóságot és rugalmasságot kínál,
és a Dante hálózaton keresztül könnyen integrálható a már meglévő rendszerbe. A jelenlegi
Shure ULXD vezeték nélküli mikrofon rendszerünk is kiváló minőségű, azonban sajnálatos módon 
csak 48 kHz-es mintavételezési frekvenciával rendelkezik, míg az Axient Digital rendszer 96 kHz-es
mintavételezési frekvenciával rendelkezik. Ezáltal a hangminőség jelentősen javulna a rendszerben,
valamint így integrálhatóvá válna direkt módon a Dante hálózaton keresztül a keverőbe, így
nem lenne szükség a hangot analóg jelekké alakítani, majd vissza. Ezáltal a hangminőség még 
kevésbé romlik.
Mint ez eddigi Dante eszközünket, ezt is a Dante Controller segítségével könnyen konfigurálhatjuk, akár egy fixen
kijelölt csatornatartományban, amelyet a keverőpulton is könnyen beállíthatunk. Így mikor rendezvényre érkezünk,
plug and play módon azonnal használhatjuk a mikrofonokat, nem kell a csatornákat újra össze párosítani.
A Shure Wireless Workbench segítségével a mikrofonokat könnyen monitorozhatjuk, és a rendszer
állapotát is ellenőrizhetjük távolról is. A rendszer bővítése jelentős költségekkel jár, így a
bővítés előtt alaposan mérlegelni kell, hogy valóban szükséges-e.
%----------------------------------------------------------------------------
\begin
    {figure}[H]
    \centering
    \includegraphics[width=70mm, keepaspectratio]{figures/axient_digital.png}
    \caption{Shure Axient Digital vezeték nélküli mikrofon rendszer}
    \label{fig:shure_axient_digital}
\end{figure}
%----------------------------------------------------------------------------
\begin{comment}
    
\subsubsection{TASCAM multitrack recorder} %TASCAM – DA-6400
%----------------------------------------------------------------------------
Egy másik fejlesztési lehetőség a rendszerben egy multitrack recorder beszerzése.
Ez a készülék lehetővé teszi a koncertek soksávos rögzítését, amelyeket később
visszahallgathatunk, vagy akár további feldolgozásra is továbbíthatunk. A TASCAM
DA-6400 egy 64 csatornás, 1U magas rackbe szerelhető multitrack recorder, amely
Dante hálózaton keresztül képes a hangot fogadni és rögzíteni. A felvétel
készítése során a hangot a Dante hálózaton keresztül kapja meg, így nem szükséges
a hangot analóg jelekké alakítani, majd vissza. Ezáltal a hangminőség nem romlik
a felvétel készítése során, és a felvétel készítése is egyszerűbb és gyorsabb lesz.
A felvétel készítése után a felvételt a TASCAM DA-6400-ról egy USB meghajtóra
menthetjük, majd további feldolgozásra továbbíthatjuk. 
%----------------------------------------------------------------------------
\begin
    {figure}[H]
    \centering
    \includegraphics[width=80mm, keepaspectratio]{figures/da_6400.jpg}
    \caption{TASCAM DA-6400 multitrack recorder}
    \label{fig:tascam_da_6400}
\end{figure}
%---------------------------------------------------------------------------- 
\end{comment}

\subsubsection{Allen \& Heath ME Personal Mixing System} % Előadóknak egyéni keverési lehetőség
%----------------------------------------------------------------------------
Az Allen \& Heath ME Personal Mixing System egy korszerű és kivételesen
rugalmas megoldás az előadók számára, amely lehetővé teszi az egyéni
monitorkeverést. Ez különösen fontos olyan helyzetekben,
ahol a színpadon több zenész dolgozik együtt, és mindegyikük
egyedi monitor igényekkel rendelkezik, ezek most már általában fülmonitorok.
Az Allen \& Heath ME rendszer három fő komponensből áll: a ME-U 
disztribúciós hubból, a ME-500 személyi keverőegységből, valamint a 
Dante hálózati kompatibilitást biztosító M-DANTE kártyából.
A ME-U egy 10 portos disztribúciós hub, amely lehetővé teszi több 
ME-500 keverőegység egyszerű és gyors csatlakoztatását. A rendszer egyik
kiemelkedő előnye, hogy minden csatlakoztatott eszköz egyetlen CAT5 
Ethernet kábelen keresztül kapja a tápellátást és az audiojelet is, ami
jelentősen egyszerűsíti a telepítést és az üzemeltetést. A ME-U támogatja 
a különböző digitális audió hálózati formátumokat, többek között a 
Dante protokollt is, amelyet az M-DANTE kártya biztosít.
Az M-DANTE kártya lehetővé teszi, hogy a ME rendszer integrálódjon 
a Dante alapú hálózati audió rendszerekkel.
%----------------------------------------------------------------------------
\begin{figure}[H]
    \centering
    \begin{minipage}{0.45\textwidth}
        \centering
        \includegraphics[width=50mm, keepaspectratio]{figures/me_u.jpg}
        \caption{A\&H ME-U disztribúciós hub}\label{fig:me_u}
    \end{minipage}\hfill
    \begin{minipage}{0.45\textwidth}
        \centering
        \includegraphics[width=50mm, keepaspectratio]{figures/m_dante.jpg}
        \caption{A\&H M-DANTE kártya}\label{fig:m_dante}
    \end{minipage}
\end{figure}
%----------------------------------------------------------------------------
\begin{comment}
\begin
    {figure}[H]
    \centering
    \includegraphics[width=65mm, keepaspectratio]{figures/me_500.jpg}
    \caption{Allen \& Heath ME-500 személyi keverőegység}
    \label{fig:martin_audio_wpl}
\end{figure}
\end{comment}
%----------------------------------------------------------------------------
%----------------------------------------------------------------------------
\subsubsection{Martin Audio WPL LineArray rendszer} % Nagyobb rendezvényekre
%----------------------------------------------------------------------------
Amennyiben egy sokkal nagyobb rendezvényről van szó, és a jelenlegi rendszerünk már nem
tudná lefedni a területet, akkor a rendszer bővítése mellett egy nagyobb szériás LineArray
rendszer beszerzése is szükséges lehet. A Martin Audio WPL sorozatú ládái a jelenleg
elérhető legnagyobb terméke a Wavefront Precision sorozatban. 
Ebben az esetben a fő rendszert a WPL képviselné és WPC lenne az in-out fill rendszer.
A WPM szériás ládák pedig delayként szolgálnának a terület hátsó részén.
Viszont az említett teljes rendszerbővítésnek jelentős költségei vannak, így a rendszer
bővítése előtt alaposan mérlegelni kell, megtérül-e a befektetés hosszabb távon.
%----------------------------------------------------------------------------
\begin
    {figure}[H]
    \centering
    \includegraphics[width=50mm, keepaspectratio]{figures/wpl_front_view.jpg}
    \caption{Martin Audio WPL LineArray modul}
    \label{fig:martin_audio_wpl}
\end{figure}
%----------------------------------------------------------------------------
\subsection{Bővítés nagyobb interfészre és keverőpultra}
%----------------------------------------------------------------------------
Tegyük fel, hogy egy szimfonikus zenekar koncertjét szeretnénk hangosítani, ahol a zenekar
tagjainak száma meghaladja a 64 főt és mindenki dedikált mikrofonnal rendelkezik. Ebben az
esetben a 64x64-es Dante interfész már nem elegendő, mivel a zenekar tagjainak száma
meghaladja a csatornaszámot. Ebben az esetben a rendszer bővítésére van szükség.
Így az Allen \& Heath SQ sorozatú keverőpultjai már nem elegendőek, mivel ezekbe a keverőkbe ez az interfész a maximális.
Ebben az esetben egy nagyobb csatornaszámú keverőpultot kell választanunk, amelyek közül a
az Avantis és a dLive sorozatú keverők jöhetnek szóba. Ezek a keverők már kaphatóak
128x128-as Dante interfésszel is, így megnövelve a csatornaszámot. Azonban ezek a keverők
jelentősen magasabb árkategóriába tartoznak, mint az SQ sorozatú keverők, így a bővítés
költsége is nagyobb lesz. A keverőpultok fejlesztése mellett szükséges további
stageboxokat is beszerezni az igényelt csatornaszám eléréséhez.
%----------------------------------------------------------------------------
\begin
    {figure}[H]
    \centering
    \includegraphics[width=60mm, keepaspectratio]{figures/dlive-s7000.jpg}
    \caption{Allen \& Heath dLive S7000 keverőpult}
    \label{fig:dLive_S7000}
\end{figure}
%----------------------------------------------------------------------------


% Köszönetnyilvánítás - opcionális
%~~~~~~~~~~~~~~~~~~~~~~~~~~~~~~~~~~~~~~~~~~~~~~~~~~~~~~~~~~~~~~~~~~~~~~~~~~~~~~~~~~~~~~
%----------------------------------------------------------------------------
\chapter*{\koszonetnyilvanitas}\addcontentsline{toc}{chapter}{\koszonetnyilvanitas}
%----------------------------------------------------------------------------
Szeretném őszinte hálámat és köszönetemet kifejezni belső konzulensemnek, Paál Dávidnak, aki a szakdolgozat
megírásában nyújtott felbecsülhetetlen segítségével, szakértelmével és támogatásával végigkísért.
Külső konzulensemnek, Tamás Dávidnak szintén mély hálával tartozom az általa munkámra szánt számos óráért, 
amelyeket gondos átnézésre, hasznos visszajelzések adására fordított, és hálás vagyok türelméért, valamint megértéséért. Valamint
megtiszteltetés volt számomra, hogy ez a hangrendszer a szakdolgozatom keretein belül valósulhatott meg.

Köszönetet szeretnék mondani a TéDé Rendezvényeknél dolgozó kollégáimnak is, akik minden ötletemet, 
legyen az bármilyen különleges, támogatták, és segítettek azok megvalósításában. A közös munka eredménye önmagáért beszél
a sok pozitív visszajelzés és sikeres rendezvény után.
%----------------------------------------------------------------------------
\begin {figure}[h!]
    \centering
    \includegraphics[width=0.5\textwidth]{figures/danci_wpc.jpg}
    \caption{A szakdolgozat szerzője a megépített WPC Line Array mögött az adott rendezvényen}
\end {figure}
%----------------------------------------------------------------------------
A csapattal végzett munkából szerzett tapasztalatok és élmények értékes betekintést nyújtottak a mérnöki 
szakma gyakorlati oldalába. Rávilágítottak arra, hogy a valóságban gyakran másképp alakulnak a dolgok, mint ahogy azt a tervezés során elképzeljük. 
Megtanultam alkalmazkodni, gyorsan reagálni a változásokra és kihívásokra, valamint nagy hangsúlyt fektetni a csapatmunkára. 
Emellett lehetőséget kaptam arra is, hogy szakdolgozatomhoz releváns adatokat gyűjtsek és felhasználjak, 
továbbá a sok raktári workshop nélkül a dolgozat nem jöhetett volna létre. 
Támogatásuk és bátorításuk a projekt minden szakaszában rendkívül értékes volt.

Végezetül, de nem kevésbé fontos módon, szeretném megköszönni minden barátomnak és 
családtagomnak a szakdolgozatom elkészítése során nyújtott állandó támogatásukat és bátorításukat. 
Értékes meglátásaik és építő jellegű visszajelzéseik kulcsszerepet játszottak a szakdolgozat sikerében. 
Hálás vagyok az irántam tanúsított kitartó támogatásukért és biztatásukért ezen az úton.


% Ábrák listája - a word-ös sablon szerint szükséges
%~~~~~~~~~~~~~~~~~~~~~~~~~~~~~~~~~~~~~~~~~~~~~~~~~~~~~~~~~~~~~~~~~~~~~~~~~~~~~~~~~~~~~~
\listoffigures\addcontentsline{toc}{chapter}{\listfigurename}


% Táblázatok listája - opcionális
%~~~~~~~~~~~~~~~~~~~~~~~~~~~~~~~~~~~~~~~~~~~~~~~~~~~~~~~~~~~~~~~~~~~~~~~~~~~~~~~~~~~~~~
%\listoftables\addcontentsline{toc}{chapter}{\listtablename}


% Irodalomjegyzék
%~~~~~~~~~~~~~~~~~~~~~~~~~~~~~~~~~~~~~~~~~~~~~~~~~~~~~~~~~~~~~~~~~~~~~~~~~~~~~~~~~~~~~~
\addcontentsline{toc}{chapter}{\bibname}
\bibliography{bib/mybib}


% Függelékek
%~~~~~~~~~~~~~~~~~~~~~~~~~~~~~~~~~~~~~~~~~~~~~~~~~~~~~~~~~~~~~~~~~~~~~~~~~~~~~~~~~~~~~~
\include{content/appendices}

%\label{page:last}
\end{document}